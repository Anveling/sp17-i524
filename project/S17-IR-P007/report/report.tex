\documentclass[9pt,twocolumn,twoside]{../../styles/osajnl}
\usepackage{fancyvrb}
\journal{i524} 

\title{Identifying spam messages using R and Pandas over
  Docker Swarm}

\author[1,*]{Sagar Vora}
\author[1,**]{Rahul Singh}

\affil[1]{School of Informatics and Computing, Bloomington, IN 47408, U.S.A.}

\affil[*]{Corresponding authors: vorasagar7@gmail.com}
\affil[**]{Corresponding authors: rahul\textunderscore singh919@yahoo.com}

\dates{project-1, \today}

\ociscodes{Docker, Swarm, R, Pandas}

% replace this with your url in github/gitlab
\doi{\url{https://github.com/cloudmesh/sp17-i524/raw/master/project/S17-IR-P007/report/report.pdf}}

\begin{abstract}
 
A classification model shall be built by working on a training data
set of 5574 text messages, each marked as a spam or a legitimate
message. The model shall be used to correctly predict the class of any
new incoming text message as a spam or a legitimate one. The
application shall be deployed using \cite{www-docker-swarm} Docker
Swarm while data manipulation and classification shall be done using
Pandas and R in conjunction.

\end{abstract}

\setboolean{displaycopyright}{true}

\begin{document}

\maketitle

\TODO{test}

\section{Introduction}

To address the problem of incoming spam messages, a model shall be
developed using the Bayesian Classification technique to correctly
classify each incoming email/text message as a spam or a legitimate
one. The model aims at developing a message filter that shall
correctly classify messages based on word probabilities that are
extracted from the training dataset.The training dataset to build the
model consists of 5574 message records. Dataset taken from
\cite{www-sms-spam-dataset}. The training process shall use the
cross-validation feature provided by R to build the classification
model and use Bayes theorem of conditional probability to predict the
class of each incoming message.

\section{Design}
\subsection{Building the Classification model}

\subsubsection{CrossValidation for the training data}
To develop an efficient training model, we shall partition the data
into 2 subsets - training data and classification data. We shall
choose one of the subsets for training and other for testing. In the
next iteration the roles of the subsets shall be reversed, i.e the
training data becomes the classification one and vice versa. This
operation shall be carried out until each individual record is used
both as a classification and training record. We shall use the cross
validation feature provided by R for this subsampling. This
subsampling technique handles the underfitting problem and guarantees
an effective classification model.


\subsubsection{Training process}
Content of each of the spam marked messages shall be processed through
Naive Bayes Classifier.  The classifier shall maintain a bag of words
along with the count of each word occuring in the spam messages. This
word count shall be used to calculate and store the word probability
in a table that shall be cross-referenced to determine the class of
the record on classification
data \cite{paper-classification-of-email}.

A selected few words have more probability of occuring in a spam
messages than in the legitimate ones. Eg: The word "Lottery" shall be
encountered more often in a spam message.  The classifier shall
correlate the bag of words with spam and non-spam messages and then
use Bayes Theorem to calculate a probability score that shall indicate
whether a message is a spam or not. The results shall be verified with
the results available on the training dataset and the classifier
accuracy shall be calculated.  The classifier shall use the Bayesian
theorem over the training dataset to calculate probabilities of such
words that occur more often in spam messages and later use a summation
of scores of the occurence of these word probabilities to estimate
whether a message shall be classified as spam or not. After working on
several samples of the training dataset, the classifier shall have
learned a high probability for spam based words whereas, words in
legitimate message like family member or friends names shall have a
very low probability of occurence.

\subsection{Classifying new data}

Once the training process has been completed, the posterior
probability for all the words in the new input email is computed using
Bayes theorem. A threshold value shall be defined to classify a
message into either class. A message's spam probability is computed
over all words in its body and if the sum total of the probabilities
exceeds the predefined threshold, the filter shall mark the message as
a spam \cite{www-wiki-naivebayes}.

A higher filtering accuracy shall be achieved through filtering by
looking at the message header i.e the sender's number/name. Thereby if
a message from a particular sender is repeatedly marked as spam by the
user, the classifier need not evaluate the message body if it is from
the same sender.


\section{Discussion}
TBD


\section{Deployment}
Our application will be deployed using Ansible \cite{www-ansible}
playbook. Automated deployment should happen on two or more nodes
clouds or on multiple clusters of a single cloud. Deployment script
should install all necessary software along with the project code to
the cluster nodes.

\section{Conclusion}

TBD

\section{Acknowledgement}

We acknowledge our professor Gregor von Laszewski and all associate
instructors for helping us and guiding us throughout this project.

\section{Appendices}
TBD

% Bibliography

\bibliography{references} 

\end{document}

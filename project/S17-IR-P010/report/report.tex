\documentclass[9pt,twocolumn,twoside]{../../styles/osajnl}
\usepackage{fancyvrb}
\journal{i524} 

\title{Deployment of Vehicle Detection application on Chameleon clouds}

\author[1,*]{Abhishek Naik}
\author[2,*]{Shree Govind Mishra}

\affil[1]{School of Informatics and Computing, Bloomington, IN 47408, U.S.A.}
\affil[2]{School of Informatics and Computing, Bloomington, IN 47408, U.S.A.}
\affil[*]{Corresponding authors: absnaik810@gmail.com, shremish@indiana.edu}

\dates{project-001, \today}

\ociscodes{Vehicle detection, Ansible, Cloudmesh, OpenCV, Haar Cascades, Cloud, I524}

% replace this with your url in github/gitlab
\doi{\url{https://github.com/absnaik810/sp17-i524/blob/master/project/S17-IR-P010/report.pdf}}

\begin{abstract}
This project focuses on the deployment of Vehicle Detection
application on multiple Chameleon clouds using Ansible playbook.  It
also focuses on the benchmarking of the deployment results and its
analysis.\newline
\end{abstract}

\setboolean{displaycopyright}{true}

\begin{document}

\maketitle

\section{Introduction}

Vehicle Detection forms an integral part of the development of new
technologies like fully self-driving cars, etc.  One of the techniques
to perform such detection is by using Haar Cascades \cite{haar-cascade}.  This
technique has been applied to vehicle detection by creating a
haar-cascade cars.xml file which has been trained using 526 rear-end
images of cars.  In this project, we would be extending this vehicle
detection approach to enable it to run on multiple clouds.  This
deployment would initially be done on the localhost, followed by
deployment on 1, 2, 3 and 4 clouds.  Cloudmesh client would be used
for cloud management and Ansible scripts would be used for software
stack deployment.  Appropriate benchmarking would be carried out at
each iteration using some benchmarking technique or tool (TBD).

\section{Software stack}
\begin{itemize}
\item[$\bullet$] Ansible
\item[$\bullet$] Cloudmesh
\item[$\bullet$] TBD
\end{itemize}

\section{Execution Plan}

The execution plan that would be followed is as under:

\subsection{Week 1}
Deployment of the vehicle detection application on the localhost,
using Ansible playbook.  Benchmarking of the important factors that
are chosen for observation.  Analysis of the benchmarking results thus
obtained.

\subsection{Week 2}
Reservation of a single Chameleon cloud using cloudmesh client.
Deployment of the vehicle detection application on the cloud, using
Ansible playbook.  Benchmarking of the important factors that are
chosen for observation.  Analysis of the benchmarking results thus
obtained.

\subsection{Week 3}
Reservation of 2 Chameleon clouds using cloudmesh client.  Deployment
of the vehicle detection application on the clouds, using Ansible
playbook.  Benchmarking of the important factors that are chosen for
observation.  Analysis of the benchmarking results thus obtained.

\subsection{Week 4}
Reservation of 3 and 4 Chameleon clouds using cloudmesh client.
Deployment of the vehicle detection application on the clouds, using
Ansible playbook.  Benchmarking of the important factors that are
chosen for observation.  Analysis of the benchmarking results thus
obtained.

\subsection{Week 5}
Final analysis of the benchmarking results obtained so far.  Studying
the effect of software stack deployment on multiple clouds.
Publishing the final report and conclusion.This project focuses on the
deployment of Vehicle Detection application on multiple Chameleon
clouds using Ansible playbook.  It also focuses on the benchmarking of
the deployment results and its analysis.

\section{Deployment}

As per the current plan, we would first identify the software stack
that would be required to run the vehicle detection application on the
localhost, which is a Ubuntu 16.04.02 machine with 3 GB RAM.  This
software deployment would be done using an Ansible script
\cite{Ansible}.  Ansible is an automation engine which we would use to
orchestrate the software deployment via playbooks using the
inventory.txt file.  Important factors like the time required for
deployment, the 'ease' of deployment (TBD), etc. would be benchmarked
using some benchmarking technique or tool (TBD).

The next step would be to deploy this software stack on a single
cloud.  We would be using the cloudmesh client
\cite{github-cloudmesh-client}, a command line based client, to
reserve a cloud and enable access to it.  We would consider the
feasibility of deploying the software stack on the cloud, by careful
observation of the benchmarking results obtained in the previous step
of deploying it on the localhost.  Depending upon the outcomes, we
would be able to identify the minimum system requirements that would
be typically required for the deployment of the software stack.  This
deployment would again be achieved by using Ansible
scripts \cite{Ansible}.  These results would again we benchmarked for
further analysis.
	
As the next step, we would carry out this deployment on two clouds.
Ansible would again be used for deployment of the software stack and
cloudmesh client for the management of the clouds.  The results
obtained would again be benchmarked and analysis performed.  We would
iterate this for 3 and 4 clouds.  More clouds might be considered
depending upon the time constraints of the project.

Finally the observations and analyses would be published as a report.

\section{Benchmarking results and analysis}
TBD

\section{Conclusion}
TBD

\section{Acknowledgements}
TBD

\bibliography{references}

\end{document}

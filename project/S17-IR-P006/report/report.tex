\documentclass[9pt,twocolumn,twoside]{styles/osajnl}
\usepackage{fancyvrb}
\journal{i524} 

\title{cloudmesh cmd5 extension for AWS}

\author[1]{Milind Suryawanshi}
\author[2]{Piyush Rai}

\affil[1]{School of Informatics and Computing, Bloomington, IN 47408, U.S.A.}
\affil[2]{School of Informatics and Computing, Bloomington, IN 47408, U.S.A.}

\dates{project-006, \today}

\ociscodes{Cloud, I524}

% replace this with your url in github/gitlab
\doi{\url{https://github.com/cloudmesh/sp17-i524/blob/master/project/S17-IR-P006/report/report.pdf}}


\begin{abstract}
	
 The cludmesh client will be extended to support cluster deployment on AWS using cmd5.
 
\end{abstract}

\setboolean{displaycopyright}{true}

\begin{document}

\maketitle

\section{Introduction}

We are going to look at cloudmesh client \cite{www-client} man pages, understand what features are already supported for cloud like chameleon and implement similar support for AWS. The new features will be provided under aws subcommand. We will study the technologies currently in-use by the project and analyze how they can be used further to achieve our objective.

\section{Design}

TBD

\section{Technology Used}

\renewcommand{\labelitemi}{\scriptsize$\square$} 

\begin{itemize}
	\item Cloud Mesh
	\item AWS
	\item Cmd5
	\item libcloud
\end{itemize}

\section{Steps}

\renewcommand{\labelitemi}{\scriptsize$\square$}

\begin{itemize}
	\item Understand cloudmesh client.
	\item Propose changes and get reviewed.
	\item Open aws account and test basic commands.
	\item Make changes and test.
	\item Benchmark.
\end{itemize}

\section*{Acknowledgements}

TBD


% Bibliography

\bibliography{references}
 
\section*{Author Biographies}
\begingroup
\setlength\intextsep{0pt}
\begin{minipage}{1.0\columnwidth}
  \noindent
  {\bfseries Milind Suryawanshi} received his BE (Electronics and Telecommunication) in 2010 from
  The University of Pune. His research interests also include Big Data analytics for intelligence and research. 
\end{minipage}
\begin{minipage}{1.0\columnwidth} 
  \noindent
  {\bfseries Piyush Rai} received his BE (Computer) in 2011 from
  The University of Pune. His research interests also include Big Data analytics for military intelligence and frinancial markets. 
\end{minipage}

\endgroup

\newpage

\appendix

\section{Work Breakdown}

The work on this project was distributed as follows between the
authors:

\begin{description}

\item[Milind Suryawanshi.] TBD

\item[Piyush Rai.] TBD

\end{description}


\end{document}

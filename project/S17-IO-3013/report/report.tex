\documentclass[9pt,twocolumn,twoside]{styles/osajnl}
\usepackage{fancyvrb}
\usepackage[colorinlistoftodos,prependcaption,textsize=normal]{todonotes}
\newcommand{\TODO}[2][]{\todo[color=red!10,inline,#1]{#2}}
\newcommand{\GE}{\TODO{Grammar}}
\newcommand{\SE}{\TODO{Spelling}}
\newcommand{\TE}{\TODO{Term}}
\newcommand{\CE}{\TODO{Citation}}
\journal{i524} 

\title{Proposal for Music Predictive Analysis Project based on Lyrics}

\author[1,*]{Leonard Mwangi}

\affil[1]{School of Informatics and Computing, Bloomington, IN 47408, U.S.A.}

\affil[*]{Corresponding authors: lmwangi@iu.com}

\dates{project-01, \today}

\ociscodes{Cloud, I524}

% replace this with your url in github/gitlab
\doi{\url{https://github.com/lmundia/sp17-i524/tree/master/project/S17-IO-3013/report/report.pdf}}


\begin{abstract}
Being certain that lyrics of your song will lead to the next greatest
hit would boost confidence to a lot of amateur artists who are faced
with fears of never making it thus never attempting to make good their
creativity. With Machine Learning (ML) this can be a thing of the
past, these artists would have the ability to let ML models determine
the viability of their lyrics becoming the next hit based on history
of other songs that have made it to top. Through training, the model
can certainly determine the outcome of different songs which will be
depicted in this project.\newline
\end{abstract}

\setboolean{displaycopyright}{true}

\begin{document}

\maketitle

\section{Introduction}
When faced with the decision to forward their song to a recording
company, amateur artists find it daunting due to uncertainty of
whether their song would be recorded and if it is if it will make them
wealthy.  Having ability to run the lyrics through a predictive
analysis process that would determine the viability of the song making
it would be a huge win and confidence booster to many artists. That
prediction is achievable by use of machine learning and creating a
model that takes already greatest his and trains it to determine what
makes the song successful. This would be done by analyzing the lyrics,
the locality and time of release.

In this project, we will utilize machine learning to help determine
the viability of a song becoming the next greatest hit based on the
lyrics, time of production, locality and the artist. The project will
utilize the greatest hits of all time [1] to train a model which will
then be used to analyze larger dataset of random songs [2] and provide
an in-depth analysis of the next possible hit. The project will
utilize a Hadoop cluster deployed on Chameleon Cloud using CloudMesh
to accomplish this analysis.

The following components will be utilized to accomplish the project:
\begin{itemize}
\item Ansible
\item Apache Mesos
\item Apache Spark
\item MongoDB
\item Million Song Dataset
\item Billboard charts
\item Python Scripts
\end{itemize}

\section{Components Roles}

\section*{Ansible}
Will be used to install software packages and define roles to
different nodes in the cluster.

\section*{Apache Mesos}
Will act as the scheduler for the environment.

\section*{Apache Spark}
Due to Sparks ability to parallel process, we’ll utilize it to process
the dataset to achieve the required performance while providing
in-depth analysis.

\section*{MongoDB}
MongoDB will be used as the repository for the dataset.

\section{Million Songs Dataset}
This is a freely-available community maintained dataset  \cite{www-millionsong}. The dataset
will be used by ML as the source of random songs that will be analyzed
for results. This project will utilize a subset of the dataset due to
time and size of our development environment. 

\section*{Billboard Charts}
In conjunction with Million Songs Dataset, Billboard charts  \cite{www-billboard} will be
used to determine the greatest hits of all time, which will be used to
train the model on how to determine a great hit. 

\section*{Python scripts}
Scripts will be used to train the model and determine the next
greatest hit.


\section{Conclusion}

Ability for amateur artists, artists and record labels to quick
determine the viability of a hit is paramount to their success and
missed chances due to inexperience, fear of unknows, bad song or
acting when time is not ripe can be costly. Machine learning has the
ability to change these outcomes, a well-trained model can help
determine with high accuracy where the song will end up. 


% Bibliography

\bibliography{references}
 
\end{document}

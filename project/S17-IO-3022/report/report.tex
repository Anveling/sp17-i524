\documentclass[9pt,twocolumn,twoside]{../../styles/osajnl}
\usepackage{fancyvrb}
\journal{i524} 

\title{Detection of street signs in videos in a robot swarm}

\author[1,*]{Sunanda UnnI}
\author[1,**]{Gregor von Laszewski}

\affil[1]{School of Informatics and Computing, Bloomington, IN 47408, U.S.A.}

\affil[*]{Corresponding authors: suunni@indiana.edu}
\affil[**]{Corresponding authors: laszewski@gmail.com}

\dates{project-1: Data analysis of Robot Swarm data, \today}

\ociscodes{Cloud, I524}

% replace this with your url in github/gitlab
\doi{\url{https://github.com/cloudmesh/sp17-i524/blob/master/project/S17-IO-3022/report/report.pdf}}


\begin{abstract}
Extracting and identifying traffic signals from the videos captured by Robot swarms to help in recognizing the pattern and benchmarking the performance of the setup.
\end{abstract}

\setboolean{displaycopyright}{true}

\begin{document}

\maketitle

\section{Introduction}
For test purpose we created some mobile videos of traffic in a simulated traffic setup. All saved video files are uploaded on the Hadoop HDFS \cite{www-apache-hadoop}. Batch processing is enabled on the input video files to search for key images, namely the red, green and yellow signals in the images using the OpenCV \cite{www-opencv} library's Template matching functionality. 
Hadoop Map reduce \cite{www-apache-hadoop} is used for processing and analysis of the images in the videos and getting a count of the how many red or green or yellow signals are encountered.

collectd \cite{www-collectd} is used for benchmarking of the setup with Apache Hadoop using various sized data sets and number of nodes.

\section{Technology Used}

\TODO{tables need a begin table end table}
%\begin{center}
\resizebox{\columnwidth}{!}{  
 \begin{tabular}{||c l||}
 \hline
 Technology Name & Purpose  \\ [0.5ex]
 \hline\hline
 Hadoop \cite{www-apache-hadoop} & map reduce  \\
 \hline
 OpenCV \cite{www-opencv} Pattern matching in video  \\
 \hline
 ansible \cite{www-ansible} & Automated deployment \\
 \hline
 collectd \cite{www-collectd} & Collection of statistics of setup for benchmarking \\
 \hline
\end{tabular}
}
%\end{center}


\section{Plan}

\TODO{tables need a begin table end table}
\resizebox{\columnwidth}{!}{  
\begin{tabular}{ |c|l|c| }
 \hline
Week & Work Item & Status \\
\hline
week1 & Ansible deployment script for Hadoop setup  & planned \\
week2 & Ansible deployment script for OpenCV setup  & planned \\
week3 & Creating sample videos & planned \\
week4 & OpenCV template matching script & planned \\
week5 & Deployment and test of basic setup & planned \\
week6 & Ansible deployment of collectd & planned \\
week7 & Performance measurement of setup and report creation  & planned \\
week8 & Exploring different setup & planned \\
 \hline
\end{tabular}
}



\section{Design}
TBD

\section{Deployment}
TBD
\section{Benchmarking}
TBD

\section{Discussion}
TBD

\section{Conclusion}

TBD

\section{Acknowledgement}



% Bibliography

\bibliography{references}
 
\end{document}

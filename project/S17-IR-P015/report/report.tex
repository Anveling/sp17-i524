\documentclass[9pt,twocolumn,twoside]{../../styles/osajnl}
\usepackage{fancyvrb}
\journal{i524} 

\title{Deployment of a Storm cluster}

\author[1,*]{Vasanth Methkupalli}
\author[1,**]{Ajit Balaga}


\affil[1]{School of Informatics and Computing, Bloomington, IN 47408, U.S.A.}

\affil[*]{Corresponding authors:mvasanthiiit@gmail.com}
\affil[**]{Corresponding authors: ajit.balaga@gmail.com}


\dates{project-P015, \today}

\ociscodes{Storm, Ansible, Java, Python}

% replace this with your url in github/gitlab
\doi{\url{https://github.com/cloudmesh/classes/blob/master/project/S17-IR-P015/report/report.pdf}}


\begin{abstract}
This project focuses on deployment of Apache Storm using Ansible
playbook on Chameleon Cloud VM, future systems cloud and benchmarking
of the deployment.
\newline
\end{abstract}

\setboolean{displaycopyright}{true}

\begin{document}

\maketitle



\section{Introduction}
Apache Storm is a distributed stream processing computation framework
under Apache. In this project, Storm cluster of one or more Chameleon
cloud VMs and future sytems cloud is de- ployed using Ansible playbook
and benchmarking is done to measure the time it took for deployement
using a TBD bench- marking tool\cite{www-apache1}\cite{www-storm-zookeeper}.


\section{Apache Storm}

Apache Storm is a free and open source distributed realtime
computation system. Storm makes it easy to reliably process unbounded
streams of data, doing for realtime processing what Hadoop did for
batch processing. Storm is simple, can be used with any programming
language, and is a lot of fun to use!. Storm has many use cases:
realtime analytics, online machine learning, continuous computation,
distributed RPC, ETL, and more. Storm is fast: a benchmark clocked it
at over a million tuples processed per second per node. It is
scalable, fault-tolerant, guarantees your data will be processed, and
is easy to set up and operate.Storm integrates with the queueing and
database technologies you already use. A Storm topology consumes
streams of data and processes those streams in arbitrarily complex
ways, repartitioning the streams between each stage of the computation
however needed. Read more in the tutorial\cite{www-apache1}\cite{www-wiki-storm}.


\section{Milestones}

\begin{itemize}
\item Performing Analysis on local VM
\item Deploying Storm and Hadoop on FutureSystems and Chameleon Cloud
\item Analysis on the distributed cloud environment
\item Benchmarking
\item Final update with report
\end{itemize}


\section{Technologies}

\begin{itemize}
\item Distributed Computation and Storage:- Storm
\item Development:- Python and Java
\item Deployment:- Ansible
\end{itemize}

    
\section{Deployment}
Ansible Playbook is used as the application and configuration
deployment tool. Deploying the hadoop and spark framework into the
cluster environment. Ansible will help push configurations to the
environment automatically based on playbooks written for various
configurations.\cite{www-storm-apache}\cite{www-storm-zookeeper}


\section{Benchmarking}
TBD

\bibliography{references}

\end{document}

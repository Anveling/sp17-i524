\documentclass[9pt,twocolumn,twoside]{styles/osajnl}
\usepackage{fancyvrb}
\journal{i524} 

\title{Head Count Detection Using a Docker Swarm Cluster}

\author[1]{Anurag Kumar Jain}
\author[1]{Pratik Sushil Jain}
\author[1]{Ronak Parekh}

\affil[1]{School of Informatics and Computing, Bloomington, IN 47408, U.S.A.}

\dates{project-000, \today}

\ociscodes{Cloud, I524}

% replace this with your url in github/gitlab
\doi{\url{https://github.com/cloudmesh/sp17-i524/tree/master/project/S17-IR-P011}}

\begin{abstract}
By deploying our face detection application on Docker cluster, we will
try to achieve high throughput by parellalizing processing of images
for head count task on multiple nodes each having thousand of
pictures.
\end{abstract}

\setboolean{displaycopyright}{true}

\begin{document}

\maketitle

\section{Introduction}

Counting the number of people in a image has been a challenge in the
field of computer vision \cite{www-face-detection-wikipedia}. Given a
huge number of images, finding the number of people in each image can
become a cumbersome task. We try to solve this issue using our
distributed approach using a docker swarm cluster which is a cluster
of Docker \cite{www-docker-wikipedia} engines, or nodes, where
services have to be deployed. We run the OpenCV
\cite{www-opencv-wikipedia} face detection algorithms on smaller data
in multiple nodes and obtain the head count of the people in each
picture.

\section*{Acknowledgements}

This project is undertaken as part of I524: Big Data And Open Source
Software Projects at Indiana University, Bloomington. We would like to
Prof. Gregor von Laszewski and Assosiate Instructors for their help.

% Bibliography

\bibliography{references}

\end{document}

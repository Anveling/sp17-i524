\documentclass[9pt,twocolumn,twoside]{styles/osajnl}
\usepackage{fancyvrb}
\journal{i524} 

\title{ Analysis of Pentaho}


\author[1,*]{Bhavesh Reddy Merugureddy}

\affil[1]{School of Informatics and Computing, Bloomington, IN 47408, U.S.A.}


\affil[*]{Corresponding authors: bmerugur@umail.iu.edu}

\dates{Paper 1, \today}

\ociscodes{Pentaho, Data Integration, Big Data, Community, ETL, MapReduce, SQL, Hadoop, OLAP}

% replace this with your url in github/gitlab
\doi{\url{https://github.com/cloudmesh/classes/blob/master/docs/source/format/report/report.pdf}}


\begin{abstract}
Pentaho is a leading business analytics and data integration tool that provides a qualified open source-based platform to assist a variety of big data deployments. It enables different organizations to utilize their data which helps them in delivering their services efficiently with minimum risk.Pentaho is often considered as an ideal application which can be used by businesses that desire to get the most out of their data and can also be used for embedded analytics. \newline
\end{abstract}

\setboolean{displaycopyright}{true}

\begin{document}

\maketitle

\section{Introduction}

Pentaho can be viewed as a business intelligence suite that provides
data mining, reporting, dashboarding and data integration
capabilities. Generally, organizations tend to obtain meaningful
relationships and useful information from the data present with
them. Pentaho addresses the obstacles that obstruct them from doing so
\cite{pent1}. The platform includes a wide range of tools that
analyze, explore, visualize and predict data easily which simplifies
blending the data. The sole objective of pentaho is to translate data
into value. Being an open and extensible source, pentaho provides big
data tools to extract, prepare and blend any data. Along with this,
the visualizations and analytics will help in changing the path that
the organizations follow to run their business. From spark and hadoop
to noSQL, pentaho transforms big data into big insights.

\section{Pentaho community}

Pentaho provides two different editions, community edition and
enterprise edition. As the name suggests, the enterprise edition comes
with more packages to provide addition support. The community edition
enables the developers or users to create complex solutions for the
problems pertaining to their business \cite{pent2}. The pentaho
community has a group of intellectual people and helps the users in
becoming a part of them and benefit from the open source
contributions. These open source projects are helpful in delivering
reliable and faster products which are timely tested by the
community. The community includes all users like developers, testers
and managers. Generally, the community edition platform enables the
developers to sketch their design and develop a rough version of their
product after which they can upgrade to enterprise edition for final
production.

Pentaho provides an interactive console to its users. With a few
clicks of the mouse, users are allowed to interact with new data
models and data. The platform hides the database connections and
underlying application server and provides access to various data
sources \cite{pent3}. It provides metadata management capabilities and
a dashboard to allow the administrators set security levels, monitor
servers and set user access. There are many server plugins and desktop
applications provided by pentaho.

\subsection{Server applications}

Business Intelligence platform is a basic service that provides
reports, displays dashboards, reports business rules and performs OLAP
analysis. The latest version comes with RESTful services and
re-written scheduler along with a migration system. It generally runs
in Apache java application server and can be embedded in any other
java application server \cite{pent1}. Pentaho analysis service is
another server application that is written in java which primarily
focuses on online analytical processing. It aggregates data into a
memory cache by performing read operation from data sources like
SQL. It comes with the pentaho platform in both the editions. These
are some of the server applications provided by pentaho.

\subsection{Desktop applications}

Pentaho data mining is a desktop application that searches for
patterns in data by performing knowledge analysis. All the techniques
of data mining such as classification, clustering, regression and
visualization are employed by this application along with some machine
learning algorithms. This helps the users in predicting the trends in
future. Pentaho metadata editor is an application that is used as an
abstraction layer from the underlying data sources and helps the users
in creating effective business models which can be used by other
applications in creating reports for the analytics. There are many
more useful desktop applications.
 
\subsection{Server plugins}

Some of the important server plugins are community data access and
data browser. Community data access is a pentaho server plugin that
provides a common layer on the business analytics server for an easy
data access. It runs the server by providing a REST interface and gets
back the results in various forms such as xml, csv or json. Community
data browser is a plugin that helps R in performing analytics on the
data. It does the job of supplying queries to R by using online
analytical processing browser.

\section{Data Integration}

Extract, transform and load (ETL) are the basic operations that act as
a tool for transforming data from one database and placing in other
database. These processes can be carried out in pentaho with the help
of a component called pentaho data integration, which is also referred
as kettle \cite{pent4}. The most useful functions of pentaho data
integration include massive load of data into databases, data
cleansing, migrating data between applications and integrating several
applications. It is metadata oriented and can be used as a standalone
application. The ability of transforming data is so high that the data
can be manipulated with a very few limitations. Various input and
output formats such as datasheets and text files are supported by
pentaho data integration.

The transformation process undergoes three steps, input step,
transformation step and output step. In the input step, data is
ingested \cite{pent5}. The data is then processed within pentaho data
integration and the transformed data is given out in the output
step. All these steps are carried out in parallel. The throughput of
transformation process is restricted to speed of the step which is
slowest. The slowest step is often referred as bottleneck. To improve
the performance of transformation process, two steps are run in a loop
which are, identification of the bottleneck and continued improvement
of bottleneck until it is no longer a bottleneck.

Pentaho data integration has a set of components that contribute to
its functionalities. They are spoon, kitchen, pan and carte
\cite{pent6}. Spoon can be considered as a desktop application that
creates simple and even complex extract, transform and load (ETL) jobs
without making the users write or read code. Spoon is the application
that is used for transformations and jobs with the help of editor. So,
it is the one that is used in most of the cases such as editing,
debugging or running a transformation or a job. As the transformations
are created in spoon, they can be executed with the help of a
standalone command line process called pan. It is an engine that reads
data, manipulates it and loads into various data sources. Kitchen is
another standalone command line process that for executing jobs. It
schedules different jobs to run at regular intervals. Carte provides
remote execution capabilities and a medium for setting up a remote ETL
server.

\section{Architecture}

Pentaho architecture can be considered as a set of four components
which are presentation layer, business intelligence platform, data and
application integration and third party applications. Data can be
provided to the presentation layer by reporting, analysis or process
management. This data can then be accessed through a web service,
portal or a browser \cite{pent7}. The security and repository issues
are dealt by the business intelligence platform. Data integration and
third party applications are respectively, the integration layer and
applications with database from various sources.

The architecture also includes a set of predefined layers such as data
layer, server layer and client layer. Data layer allows an application
to connect to a data source. Server layer serves as a middle layer and
several applications run on the server. Dashboards are provided to the
end users by deploying them on the server along with the required
reports. As mentioned above, a user console is provided that is used
for security and configuration purposes. Client layer is of two forms,
thin client and thick client. Thin client generally runs on a
server. Analyzer and dashboard editor can be considered as the
examples. Report designer and data integration come under thick client
which act as a standalone.

\section{Big Data Use cases}

Big data refers to humongous volumes of data being taken from multiple
data sources and put into data stores. A use case can be defined as a
technology solution for business specific challenges. Big data use
cases help in understanding the problems that big data addresses.

Cyber security analysis helps the end users such as data scientists
and security analysts in quickly detecting the threats. Cyber security
analytics allows the users to utilize most of the staff resources via
automation \cite{pent8}. It empowers the data scientists with
predictive analytics with the help of machine learning tools. It also
provides the automation of blending and reporting on a variety of
data. Pentaho platform can be utilized for data processing, data
ingestion and delivery of threat calls with minimal costs and
complexity.

Pentaho optimizes data warehouse and speeds up the development and
deployment processes. It employs a simplified process for offloading
to Hadoop. The offloaded data is usually less frequent data. Hand
coding in MapReduce jobs and SQL can be avoided by the usage of visual
integration tools. It provides access to data sources ranging from
relational to operational to NoSQL technologies. Pentaho MapReduce
helps in achieving high performance in a cluster environment. It
provides a graphical and intuitive big data integration.

Another use case identified by pentaho is the streamlined data
refinery. Pentaho data integration processes and refines different
data sets by using Hadoop as its data processing platform. It provides
modelled, delivered and published data sets to the users for visual
analytics just by a mouse click. It can be seen as an integration
process that blends huge volumes of highly diversified data. It also
supplies tools for in-cluster simplified data processing and is
regarded as a highly practical approach.

Pentaho’s big data support extends the 360-degree view to internal and
external customer related data. Customer service teams are provided
with time-sensitive and blended streams of data. This helps in making
profitable decisions. The presence of an adaptive big data layer
relieves several organizations from evolving technologies. Customers
are given access to customizable, intuitive and interactive
dashboards. Data scientists are provided with predictive analytics and
data mining tools.

Monetizing the data is the final use case addressed by pentaho. It
allows the users to capitalize on big data with the help of powerful
data processing and embeddable data analytics \cite{pent9}. Pentaho’s
big data analytics platform empowers easy big data ingestion and
transformation as it works as a no-code data integration
environment. It is a flexible platform that supports security and
deployments specific to customers.

\section{Comparison}

Pentaho products compete with some big names in current field such as
SAP, IBM and oracle. Pentaho provides open source solutions and is
considered to be much cheaper than the proprietary
equivalents. Jaspersoft is an established open source rival of
pentaho. Though both pentaho and jaspersoft offer similar features
with similar costs, pentaho has got wider online presence and more
followers in social media \cite{pent10}.

\section{Conclusion}

Pentaho is an open source based platform for diverse big data
deployments. It empowers analytics in any environment by delivering
governed data. It has unified data integration and analytics
components which are comprehensive and embeddable. The primary aim of
pentaho is to enable organizations to find new revenue streams with
extraordinary service at minimum risk. It helps them in harnessing the
value from their data in order to make their operations efficient and
consistent.



\bibliography{references}
 

\newpage

\appendix


\end{document}

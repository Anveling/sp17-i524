\documentclass[9pt,twocolumn,twoside]{styles/osajnl}
\usepackage{fancyvrb}
\journal{i524} 

\title{Ansible}

\author[1]{Anurag Kumar Jain}
\author[1,*]{Gregor von Laszewski}

\affil[1]{School of Informatics and Computing, Bloomington, IN 47408, U.S.A.}

\affil[*]{Corresponding authors: laszewski@gmail.com}

\dates{Paper 1, \today}

\ociscodes{Cloud, Ansible, Playbook, Roles}

% replace this with your url in github/gitlab
\doi{\url{https://github.com/anurag2301/sp17-i524}}


\begin{abstract}
Ansible is a powerful open source automation tool which can be used in
configuration management, deployment, and orchestration
tool \cite{www-ansible-wikipedia}. This paper gives an in-depth
overview of Ansible.
\newline
\end{abstract}

\setboolean{displaycopyright}{true}

\begin{document}

\maketitle

\section{Introduction}

Ansible is an open source automation engine which can be used to
automate cloud provisioning, configuration management, and application
deployment. It is designed to be minimal in nature, secure,
consistent, and highly reliable \cite{www-ansible3}. In many respects,
it is unique from other management tools and aims to provide large
productivity gains. It has an extremely low learning curve and seeks
to solve major unsolved IT challenges under a single banner.  Michael
DeHaan developed the Ansible platform and Ansible, Inc. was the
company set up to commercially support and sponsor Ansible. The
company was later acquired by Red Hat Inc. It is available on Fedora,
Red Hat Enterprise Linux, CentOS and other operating
systems \cite{www-ansible-wikipedia}.

\section{Architecture}

One of the primary differences between Ansible and many other tools in
the space is its architecture. Ansible is an agentless tool, it
doesn't requires any software to be installed on the remote machines
to make them manageable. By default is manages remote machines over
SSH or WinRM, which are natively present on those
platforms \cite{www-ansible}.

Like many other configuration management software, Ansible
distinguishes two types of servers: controlling machines and
nodes. Ansible uses a single controlling machine where the
orchestration begins. Nodes are managed by a controlling machine over
SSH. The location of the nodes are described by the inventory of the
controlling machine \cite{www-ansible3}.

Modules are deployed by Ansible over SSH. These modules are
temporarily stored in the nodes and communicate with the controlling
machine through a JSON protocol over the standard
output.\cite{www-ansible}

The design goals of Ansible includes consistency, high reliability,
low learning curve, security and to be minimalistic in nature. The
security comes from the fact that Ansible doesn't requires users to
deploy agents on nodes and manages remote machines using SSH or
WinRM. Ansible doesn't require dedicated users or credentials - it
respects the credentials that the user supplies when running
Ansible. Similarly, Ansible does not require administrator access, it
leverages sudo, su, and other privilege escalation methods on request
when necessary \cite{github-ansible}. If needed, Ansible can connect
with LDAP, Kerberos, and other centralized authentication management
systems \cite{www-ansible2}.

\section{Advanced Features}

The Ansible Playbook language includes a variety of features which
allow complex automation flow, this includes conditional execution of
tasks, the ability to gather variables and information from the remote
system, ability to spawn asynchronous long running actions, ability to
operate in either a push or pull configuration, it also includes a
“check” mode to test for pending changes without applying change, and
the ability to tag certain plays and tasks so that only certain parts
of configuration can be applied \cite{www-ansible3}. These features allow your
applications and environments to be modelled simply and easily, in a
logical framework that is easily understood not just by the automation
developer, but by anyone from developers to operators to CIOs. Ansible
has low overhead and is much smaller when compared to other tools like
Puppet \cite{www-ansible4}.

\section{Playbook and Roles}

Playbook is what Ansible uses to perform automation and
orchestration. They are Ansible's configuration, deployment and
orchestration language. They can be used to describe policy you need
your remote systems to enforce, or a set of steps in a general IT
process \cite{www-ansible5}.

At a basic level, playbooks can be used to manage configurations and
deployments to remote machines. While at an advanced level, they can
be utilized to sequence multi-tier rollouts which involves rolling
updates, and can also be used to delegate actions to other hosts,
interacting with monitoring servers and load balancers at the same time.

Playbooks consists of series of ‘plays’ which are used to define
automation across a set of hosts, known as the ‘inventory’. These
‘play’ generally consists of multiple ‘tasks,’ that can select one,
many, or all of the hosts in the inventory where each task is a call
to an Ansible module - a small piece of code for doing a specific
task. The tasks may be complex, such as spinning up an entire cloud
formation infrastructure in Amazon EC2. Ansible includes hundreds of
modules which help it perform a vast range of
tasks \cite{www-ansible}.

Similar to many other languages Ansible supports encapsulating
Playbook tasks into reusable units called ‘roles.’ These roles can be
used to easily apply common configurations in different scenarios,
such as having a common web server configuration role which can be
used in development, test, and production automation. The Ansible
Galaxy community site contains thousands of customizable rules that
can be reused used to build Playbooks.

\section{Complex Orchestration Using Playbooks}

Playbook can be used to combine multiple tasks to achieve complex
automation \cite{www-ansible5}. Playbook and Ansible can be easily used
in implementing a cluster-wide rolling update that consists of
consulting a configuration/settings repository for information about
the involved servers, configuring the base OS on all machines and
enforcing desired state. It can also be used in signaling the
monitoring system of an outage window prior to bringing the servers
offline and signaling load balancers to take the application servers
out of a load balanced pool. The usage doesn't ends here and it can be
utilized to start and stop server, running appropriate tests on the
new server or even deploying/updating the server code, data, and
content. Ansible can also be used to send email reports and as a
logging tool \cite{www-ansible}.

\section{Extensibility}

Tasks in Ansible are performed by Ansible ‘modules’ which are pieces
of code that run on remote hosts. Although there are a vast set of
modules which cover a lot of things which a user may require there
might be a need to implement a new module to handle some portion of IT
infrastructure. Ansible makes it simpler and by allowing the modules
to be written in any language, with the only requirement that they are
required to take JSON as input and product JSON as
output \cite{www-ansible}.

We can also extend Ansible through it's dynamic inventory which allows
Ansible Playbooks to run against machines and infrastructure
discovered during runtime. Out of the box Ansible includes support for
all major cloud providers, it can also be easily extended to add
support for new providers and new sources as needed.

\section{One Tool To Do It All}

Ansible is designed to make IT configurations and processes both
simple to read or write, the code is human readable and can be read
even by those who are not trained in reading those
configurations. Ansible is different from genrally used software
programming languages, it uses basic textual descriptions of desired
states and processes, while being neutral to the types of processes
described. Ansible’s simple, task-based nature makes it unique and it
can be applied to a variety of use cases which includes configuration
management, application deployment, orchestration and need basis test
execution.

Ansible combines these approaches into a single tool. This not only
allows for integrating multiple disparate processes into a single
framework for easy training, education, and operation, but also means
that Ansible seamlessly fits in with other tools. It can be used for
any of the above mentioned use cases without modifying existing
infrastructure that may already be in use.

\section{Integration Of Cloud And Infrastructure}

Ansible can easily deploy workloads to a variety of public and
on-premise cloud environments. This capability includes cloud
providers such as Amazon Web Services, Microsoft Azure, Rackspace, and
Google Compute Engine, and local infrastructure such as VMware,
OpenStack, and CloudStack. This includes not just compute
provisioning, but storage and networks as well and the capability
doesn't ends here. As noted, further integrations are easy, and more
are added to each new Ansible release. And as Ansible is open source
anyone can make his/her contributions \cite{www-ansible}.

\section{Automating Network}

Ansible can easily automate traditional IT server and software
installations, but it strives to automate IT infrastructure entirely,
including areas which are not covered by traditional IT automation
tools. Due to the Ansible’s agentless, task-based nature it can easily
be utilized in the networking space, and the inbuilt support is
included with Ansible for automating networking from major vendors
such as Cisco, Juniper, Hewlett Packard Enterprise, Cumulus and
more \cite{www-ansible3}.

By harnessing this networking support, network automation is no longer
a task required to be done a separate team, with separate tools, and
separate processes. It can easily be done by the same tools and
processes used by other automation procedures you already
have. Ansible tasks also include configuring switching and VLAN for a
new service. So now users don't need a ticket filed whenever a new
service comes in.

\section{Conclusion}

Ansible is an open source community project sponsored by Red Hat. It
is one of the easiest way to automate IT. Ansible code is easy to read
and write and the commands are in human readable format. The power of
Ansible language can be utilized across entire IT teams – from systems
and network administrators to developers and managers. Ansible
includes thousands of reusable modules which makes it even more user
friendly and users can write new modules in any language which makes
it flexible too. Ansible by Red Hat provides enterprise-ready
solutions to automate your entire application lifecycle – from servers
to clouds to containers and everything in between. Ansible Tower by
Red Hat is a commercial offering that helps teams manage complex
multi-tier deployments by adding control, knowledge, and delegation to
Ansible-powered environments.

% Bibliography

\bibliography{references}
 
\end{document}

\documentclass[9pt,twocolumn,twoside]{styles/osajnl}
\usepackage{fancyvrb}
\journal{i524} 

\title{Docker(Machine,Swarm)}

\author[1]{Shree Govind Mishra}

\affil[1]{School of Informatics and Computing, Bloomington, IN 47408, U.S.A.}

\affil[*]{Corresponding authors:shremish@indiana.edu and govindmisharaalld@gmail.com}
\dates{project-000, \today}

\ociscodes{Cloud, I524, Big Data, Virtualisation, Docker, Docker Swarm, Docker Machine , Virtual Machines, VMs, Containers,LInux Containers, lightweight containers,LXC}

% replace this with your url in github/gitlab
\doi{\url{https://github.com/argetlam115/sp17-i524/blob/master/paper1/S17-IR-2021/report.pdf}}

\begin{abstract}
  Docker is a container based technology which helps in the packaging
  and shipping of applications quickly and across networks. Docker is
  being well accepted and appreciated in the Agile Software
  Development workforce and among the DevOps because of their
  attributes of Continous Integeration and testing.To create a cluster
  of Docker Engines into a single docker virtual engine is used
  Docker-Swarm sed which is tested for more than 50,000
  containers. 
\end{abstract}

\setboolean{displaycopyright}{true}

\begin{document}

\maketitle


\section{Introduction}
\cite{www-docker-2}Docker is an open-source container based
technology. It is an extension of Linux Containers (LXC): which is a
unique kind of lightweight, application-centric virtualization that
drastically reduces overhead and makes it easier to deploy software on
servers. A container allows a developer to package up an application
and all its part includig it's code, stack it runs on, dependencies it
is associated with,system tools and everything the application requirs
to run within an isolated enviorment. This makes it easier for
programmers and developers to run more apps on the same server and it
even makes it easier to package and ship the apps very frequently.
Docker has been able to popularize the container approach in part
because it’s improved the security and simplicity of container
environments. Plus, interoperability is enhanced by its association
with major companies – such as Google, Canonical, and Red Hat – on its
open source element libcontainer.

\section{How Docker differs from VMs}

The virtualisation of application can be obtained with Hyervisors or
Virtual Machine Manager(VMM) which makes it easy for applications to
the run in isolation with one another while sharing the same
underlying hardware articture. VM hypervisors, such as Hyper-V, KVM,
and Xen, all are "based on emulating virtual hardware" which means
they’re fat in terms of system requirements.Instead of virtualizing
hardware, containers rest on top of a single Linux instance.A full
virtualized system gets its own set of resources allocated to it, and
does a very minimal sharing of thoses resources. So, the developer
gets more isolation but each Instance of the VM aslo has many "Junk VM
files", which is not useful to the developers as it gets "heavy".Since
they only require to keep the virtal machine image and it can be kept
small.Thus, the smaller the image is the less we need to store and the
less you need to send around the network which makes them fairly
lightwight in comparison to the Virtal Machines. Thus, we can run even
use 1000s of containers on a same host OS.
\cite{www-stackoverflow-docker}A full virtualized system usually
takes time to start whereas the Docker contaier do take even less than
seconds to up start and running with a lower overhead than the VMs.
Containers are potentially much more efficient than VMs because
they’re able to share a single kernel and share application libraries.
This can lead to substantially smaller RAM footprints even when
compared to virtualisation systems that can make use of RAM
overcommitment.  Storage footprints can also be reduced where deployed
containers share underlying image layers. IBM’s Boden Russel has done


\section{Uses Of Docker}

\subsection{Docker for DevOps}
Another major use of Docker is it's use in the DevOps community. Docker 
is not there to replace other configuration management tools and instead 
can be incoprporated with other configuration management tools like 
Chef, Puppet, Salt or ansible. The other major benifit of using 
Docker, Dockerfiles, the registry amd the whole Docker ecosystem is that 
the teams don't have to learn domain specific language as these are 
easier to learn than the domain specific Ecosystems. Though many a times 
Docker can be made to work with the other configuration management 
tools.\cite{www-docker-1}Docker can also the eliminate the need for a 
development team to have the same versions of everything installed on 
their local machine. The repeatable nature of Docker images makes it 
easier for them to standardize their production code and configurations.
Their work has led to the creation of Helios, an application that manages
Docker deployments across multiple servers and that alerts them when a
server isn't running the correct version of a container.

\subsection{Docker as Virtualized Sandbox}
\cite{www-docker-1} Docker allows sytems administrators and developers
to build applications that can berun on any Linux distribution or
hardware in a virtualized sandbox without making custom builds for all
the different enviorments.Finally, it’s easy to deploy Docker
containers in a cloud scenario. You can easily integrate it with
typical DevOps environments seamlessly (Ansible, Puppet, etc.) or use
it as a standalone.

\subsection{Docker for continuous integration}
Docker can be used as git for continious integeration. Docker is
similar as changes in the system can be tracked just like chaneges in
the git.  These collaboration features (docker push and docker pull)
are one of the most disruptive parts of Docker. The fact that any
Docker image can run on any machine running Docker is very much
appreciated. But The Docker pull/push are the too new fot the first
time developers and ops guys have ever been able to easily collaborate
quickly on building infrastructure together.\cite{www-docker-2}The app
guys can share app containers with ops guys and the ops guys can share
MySQL and PosgreSQL and Redis servers with app guys.




\section{Docker MACHINE}

\cite{www-docker-machine}Docker Machine is a tool that lets you install 
Docker Engine on virtual hosts, and manage the hosts with docker-machine 
commands. You can use Machine to create Docker hosts on your local Mac 
or Windows box, on your company network, in your data center, or on 
cloud providers like AWS or Digital Ocean. Docker Machine is the only 
way to run Docker Engine on Mac or Windows previous to Dockerv1.12 
and starting with the Dockerv1.12 we have Docker for Mac and Docker for 
Windows. Thus we can create a cluster of Docker hosts which is called
a Swarm using Docker Machine.

\section{Docker SWARM}
\cite{www-docker-swarm}Docker Swarm provides native clustering 
capabilites to turn a group of Docker engines into a single, vrtual 
Docker Engine. These help you scale up the applications as if these all 
are running on a single,huge computer.It does so by
providing a stabdard Docker API where if any tool which communicates
with the Docker deamon can use Docker Swarm to transparently scale to
multiple hosts: Dokku, Docker Compose, Krane, Flynn, Deis, DockerUI,
Shipyard, Drone, Jenkins and, of course, the Docker client itself.
Docker Networking, Volumes and plugins can also be used through their
respective Docker commands via Swarm. Swarm has been tested and is
production ready to scale up to one thousand (1,000) nodes and fifty
thousand (50,000) containers with no performance degradation in
spinning up incremental containers onto the node cluste Swarm also
comes with a built-in scheduler, but you can easily plugin the Mesos
or Kubernetes backend while still using the Docker client for a
consistent developer experience. To find nodes in your cluster, Docker
Swarm can use either a hosted discovery service, static file, etcd,
consul and zookeeper depending on what is best suited for your
environment.


\section{Ecosystem Support to Docker}
\cite{www-docker-1}Finally, it’s easy to deploy Docker containers in a cloud
scenario. You can easily integrate it with typical DevOps environments
seamlessly (Ansible, Puppet, etc.) or use it as a standalone. The main
reason it’s so popular is simplification, says Ben Lloyd Pearson via
opensource.com. You can do local development within a system that is
identical to a live server; deploy various development environments
from your host that each use their own software, OS, and settings;
easily run tests on various servers; and create an identical set of
configurations, so that collaborative work isn’t ever hindered by
parameters of the local host.Ecosystem support for Docker is
improving with every passing day as it is gaing the popularity among
the developers and the system operators.

Operating Systems suppport to Docker: Is compatible with virtually
any distribution with a 2.6.32 + kernel RedHat Docker collaboration to
work with across fedora and other(2.6.32)+.

It is compatible with Private PaaS(Platform as a Service) technolgies:
OPen Shift, Solum(Rackspace and OPenstack).

Public PaaS technologies like Voxoz, Cocaine(Yandex),Biadu,etc.

Ecosystem Support for IaaS is present via Rackspace,Digital Ocean,
AWS(Amazon Web Services), AMI,etc.

Orchestration tools Support and Integeration with:
Chef,Puppt,Jerkins,Travis,Ansible..

In Open Stack Docker integeration into NOVA(compatibility with
Glance,Horizon)are aslo present

\section{Shortfalls of Docker}

Though dockers are leagcy virtualization techniques, they are not a go
to solution for all kinds of virtualization and cannot be considered
as a replacement of the VMs.

VMs are self constarined with as they have a unique operating
system(OS), drivers and application components whereas the containers
are dependent as containers under Docker cannot run on Windows server.

VMs provide high level of isolation as the system's underlying
hardware resources are all virtualized and thus any bugs or viruses
could not affetc the other VM in the virtualized enviorment.

The kinds of tools and technologies required to manage the containers
are still lacking in the industry and thus only a few management tools
from companies like Google and Docker like Kubernetes and SWarm
respectively are present which is not a good situation for an
open-source products

\section{Future with Docker}

Docker Inc has set a clear path on the development of core
capabilities (libcontainer), cross service management (libswarm) and
messaging between containers (libchan). Docker's key open source
project is libcontainers ,which is on it's way to becomiing the
default standard for Linux-based technology.  Libcontainers enables
the containers to work with the Linux namespace, control groups,
capabilities, AppArmor security profiles, network interfaces and
firewalling rules in a consistent and predictable way .
\cite{www-docker-1}Libcontainers is getting help from Google, RedHat,
and Parallels to build the program asthey will work with docker as
core maintainers of the code.  libconatiner which is written natively
in Google's Go, is also being ported into other languages. Microsoft
may be porting it to ASP.NET.  Parallels' libct, which includes
libcontainer's functionality, has native C/C++ and Python
bindings. Microsoft is even jumping on board by bringing Docker to
their Azure platform, a development that could potentially make
integration of Linux applications with Microsoft products easier than
ever before.



\section{Conclusion}

\cite{www-docker-3}With more than 1200 Docker Contributors , 10,000 Dockerized
Applications at index.docker.io 3 to 4 million Developers using
Docker, 300 Million Downloads, 32,0000 Docker related Projects and 70
percent of enterprise using DockerIt is not an exaggeration that with
Docker Ecosystem there is a greater potential for some advanced
deployment tools that combine containers, configuration management,
continuous integration, continuous delivery and service orchestration
in the days ahead.


\bibliography{references}

\end{document}


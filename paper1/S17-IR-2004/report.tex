\documentclass[9pt,twocolumn,twoside]{styles/osajnl}
\usepackage{fancyvrb}
\journal{i524} 

\title{Apache Samza}

\author[1]{Ajit Balaga, S17-IR-2004}

\affil[1]{School of Informatics and Computing, Bloomington, IN 47408, U.S.A.}

\affil[*]{Corresponding authors: abalaga@iu.edu, ajit.balaga@gmail.com}

\dates{project-000, \today}

\ociscodes{Cloud, I524}

% replace this with your url in github/gitlab
\doi{\url{https://github.com/argetlam115/sp17-i524/blob/master/paper1/S17-IR-2004/report.pdf}}

\begin{abstract}
Apache Samza is a stream processing framework which enables usesrs to analyze
streaming datawith ease. The concepts required streaming data and the 
architecture behind Samza is presented here. Samza relies on Kafka and YARN, and
its relationship with the two technologies is elaborated upon. The API is
introduced as a precursor to future streaming projects. \newline
\end{abstract}

\setboolean{displaycopyright}{true}

\begin{document}

\maketitle

\section{Introduction}
Messaging systems are a method of implementing realtime asynchronous
computation, where messages can be added to a message queue, pub-sub system, or
log aggregation system whenever an event occurs. Consumers or subscribers read
these messages from the queues or systems and take actions based on the message
contents. A messaging system stores messages and waits for consumers to consume
them. Samza, a stream processing system, is a higher level of abstraction on top
of messaging systems.\cite{www-samza}

\section{Samza}
Samza uses YARN and Kafka to provide a framework for stage-wise stream
processing and partitioning. Samza is a stream processing framework with the
following features:\cite{www-samza}

\begin{itemize}
\renewcommand{\labelitemi}{\scriptsize$\square$} 
\item Simple API
\item Managed state
\item Fault tolerance
\item Durability
\item Scalability
\item Pluggable
\item Processor isolation
\end{itemize}

The Samza client uses YARN to run a Samza job: YARN starts and supervises one or
more SamzaContainers, and your processing code runs inside those containers. The
input and output for the Samza StreamTasks come from Kafka brokers that are
co-located on the same machines as the YARN NMs.\cite{paper1}
Samza is made up of three layers:

\begin{itemize}
\renewcommand{\labelitemi}{\scriptsize$\square$} 
\item A streaming layer
\item An execution layer
\item A processing layer
\end{itemize}

Samza provides out of the box support for all three layers.

\begin{itemize}
\renewcommand{\labelitemi}{\scriptsize$\square$} 
\item Streaming: Kafka
\item Execution: YARN
\item Processing: Samza API
\end{itemize}

%Figure \ref{fig:samza-arch2} shows Samza architecture.
%\begin{figure}[htbp]
%\centering
%\fbox{\includegraphics[width=\linewidth]{images/samza3}}
%\caption{Samza Architecture 1}
%\label{fig:samza-arch2}
%\end{figure}

%These three pieces fit together to form Samza:
%Figure \ref{fig:samza-arch} shows Samza architecture.
%\begin{figure}[htbp]
%\centering
%\fbox{\includegraphics[width=\linewidth]{images/samza1}}
%\caption{Samza Architecture 2}
%\label{fig:samza-arch}
%\end{figure}

%This architecture follows a similar pattern to Hadoop:
%Figure \ref{fig:hadoop-arch} shows Hadoop architecture.
%\begin{figure}[htbp]
%\centering
%\fbox{\includegraphics[width=\linewidth]{images/samza2}}
%\caption{Hadoop Architecture}
%\label{fig:hadoop-arch}
%\end{figure}

Samza’s execution and streaming layers are differentiable from the processing
layer allowing developers to implement alternatives of their choice.

\section{Streaming with Kafka}
Kafka is a distributed pub/sub and message queueing system that provides
at-least once messaging guarantees, and highly available partitions. Each topic
(streams in Kafka) is partitioned and replicated across multiple machines called
brokers. When a producer sends a message to a topic, it provides a key, which is
used to determine which partition the message should be sent to. The Kafka
brokers receive and store the messages that the producer sends. Kafka consumers
can then read from a topic by subscribing to messages on all partitions of a
topic.\cite{paper2}
Kafka has some interesting properties:

\begin{itemize}
\renewcommand{\labelitemi}{\scriptsize$\square$} 
\item All messages with the same key are guaranteed to be in the same topic partition.
\item A topic partition is a sequence of messages in order of arrival and are stored
using a monotonically increasing offset. The consumer can keep track of the
offset by storing the offset of the last message it has processed. 
\end{itemize}

\section{Resource Management with YARN}
YARN is Hadoop’s next-generation cluster scheduler which allows you to allocate
a number of containers in a cluster of machines, and execute arbitrary commands
on them. Samza uses YARN to manage deployment, fault tolerance, logging,
resource isolation, security, and locality. YARN has three important pieces: a
ResourceManager, a NodeManager, and an ApplicationMaster. In a YARN grid, every
machine runs a NodeManager, which is responsible for launching processes on that
machine. A ResourceManager talks to all of the NodeManagers to tell them what to
run. Applications talk to the ResourceManager when they wish to run something on
the cluster. The third piece, the ApplicationMaster, is actually
application-specific code that runs in the YARN cluster. It’s responsible for
managing the application’s workload, asking for containers, and handling
notifications when one of its containers fails.\cite{paper3}
Samza provides a YARN ApplicationMaster and a YARN job runner out of the box.The
Samza client talks to the YARN RM when it wants to start a new Samza job. The
YARN RM talks to a YARN NM to allocate space on the cluster for Samza’s
ApplicationMaster. Once the NM allocates space, it starts the Samza AM. After
the Samza AM starts, it asks the YARN RM for one or more YARN containers to run
SamzaContainers. Again, the RM works with NMs to allocate space for the
containers. Once the space has been allocated, the NMs start the Samza
containers.\cite{paper4}

\section{Related Concepts}

\subsection{Streams}
A stream is composed of immutable messages of a similar type or category.
Messages can be appended to a stream or read from a stream. A stream can have
any number of consumers, and reading from a stream doesn’t delete the message.
Messages can optionally have an associated key which is used for partitioning.
\cite{www-samza}

\subsection{Jobs}
A Samza job performs a logical transformation on a set of input streams to
append output messages to set of output streams. In order to scale the
throughput of the stream processor, streams and jobs are cut into smaller units
of parallelism: partitions and tasks.\cite{www-samza}

\subsection{Partitions}
Each stream is broken into a number of partitions, each with a few properties.
Each partition is an ordered sequence of messages with an identifier called the
offset, unique to its partition. When a message is appended to a stream, it is
appended to only one of the stream’s partitions.\cite{www-samza}

\subsection{Tasks}
A job is broken into multiple tasks. Each task consumes data from a partition. A
task processes messages from each of its input partitions sequentially, in the
order of message offset, in parallel for all partitions available. The YARN
scheduler assigns each task to a machine, so the job as a whole can be
distributed across many machines. The number of tasks in a job is determined by
the number of input partitions (there cannot be more tasks than input
partitions, or there would be some tasks with no input). However, you can change
the computational resources assigned to the job (the amount of memory, number of
CPU cores, etc.) to satisfy the job’s needs. See notes on containers below. The
assignment of partitions to tasks never changes: if a task is on a machine that
fails, the task is restarted elsewhere, still consuming the same stream
partitions.\cite{www-samza}\cite{paper5}

\subsection{Dataflow Graphs}
We can compose multiple jobs to create a dataflow graph, where the edges are
streams containing data, and the nodes are jobs performing transformations. This
composition is done purely through the streams the jobs take as input and
output. The jobs are otherwise totally decoupled: they need not be implemented
in the same code base, and adding, removing, or restarting a downstream job will
not impact an upstream job. These graphs are often acyclic—that is, data usually
doesn’t flow from a job, through other jobs, back to itself. However, it is
possible to create cyclic graphs if you need to.\cite{www-samza}

\subsection{Containers}
Partitions and tasks are both logical units of parallelism—they don’t correspond
to any particular assignment of computational resources. Containers are the unit
of physical parallelism, and a container is essentially a Unix process. Each
container runs one or more tasks. The number of tasks is determined
automatically from the number of partitions in the input and is fixed, but the
number of containers is specified by the user at run time and can be changed at
any time.\cite{www-samza}

\section{API}
When writing a stream processor for Samza, you must implement the StreamTask
interface. When you run your job, Samza will create several instances of your
class. These task instances process the messages in the input streams. For each
message that Samza receives from the task’s input streams, the process method is
called. The envelope contains three things of importance: the message, the key,
and the stream that the message came from. The key and value are declared as
Object, and need to be cast to the correct type. If you don’t configure a
serializer/deserializer, they are typically Java byte arrays. A deserializer
can convert these bytes into any other type.\cite{www-samza}
The getSystemStreamPartition() method returns a SystemStreamPartition object,
which tells you where the message came from. It consists of three parts:

\begin{itemize}
\renewcommand{\labelitemi}{\scriptsize$\square$} 
\item The system: the name of the system from which the message came, as defined
in your job configuration. You can have multiple systems for input and/or
output, each with a different name.
\item The stream name: the name of the stream (topic, queue) within the source
system. This is also defined in the job configuration.
\item The partition: a stream is normally split into several partitions, and
each partition is assigned to one StreamTask instance by Samza.
\end{itemize}

The API looks like this:\newline

/** A triple of system name, stream name and partition. */\newline
public class SystemStreamPartition extends SystemStream {\newline
  /** The name of the system which provides this stream. It is
      defined in the Samza job's configuration. */\newline
  public String getSystem() { ... }\newline
  /** The name of the stream/topic/queue within the system. */\newline
  public String getStream() { ... }\newline
  /** The partition within the stream. */\newline
  public Partition getPartition() { ... }\newline
}

\section{Conclusion}
Samza is a very powerful tool to work on streaming data. With its simple
approach, it allows us to analyze large amounts of streaming data on the go. The
architecture allows the developer to utilize their own resource manager and
their message handling system. Samza's architecture is very similar to hadoop,
enabling users to get started with their applications quickly and making the
learning curve shallow. With its simpleapi, Samza is a comfortable technology
for analyzing streaming data.

\bibliography{references}

\end{document}

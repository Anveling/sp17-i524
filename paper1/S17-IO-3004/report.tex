\documentclass[9pt,twocolumn,twoside]{../../styles/osajnl}
\usepackage{fancyvrb}
\journal{i524} 

\title{MySQL}

\author[1,*, +]{Cory Coulter}

\affil[1]{School of Informatics and Computing, Bloomington, IN 47408, U.S.A.}

\affil[*]{Corresponding Authors: cacoulte11@gmail.com}

\affil[+]{HID - S17-IO-3004}

\dates{project-000, \today}

\ociscodes{Database, Relational Database, Relational Database Management System, Structured Query LanguageL}

% replace this with url in github/gitlab
\doi{\url{https://github.com/cacoulte/sp17-i524/blob/master/paper1/S17-IO-3004/report.pdf}}


\begin{abstract}
This paper covers various aspects relating to the MySQL. Topics
covered include a basic overview of the technology, modern use cases
in the area of Big Data, infrastructure needs, and further educational
resources.
\end{abstract}

\setboolean{displaycopyright}{true}

\begin{document}

\maketitle

\section{Introduction}
MySQL is an "...Open Source SQL database management system [that] is
developed, distributed, and supported by Oracle Corporation."
\cite{devmysql} MySQL Server provides the means to manage data stored
in a relational database. \cite{devmysql} The origins of MySQL can be
traced back to 1979 when Micharl Widenius began creating an in-house
tool for managing databases. \cite{mik}

\section{Database}
A database provides a logical way to store information that can later
be retrieved. \cite{mysqlbible} Databases also provide a structure to
data that can facilitate the access to stored data.

\section{Relational Database}
Relational databases, which were invented by E.F. Codd in 1970, are a
type of database that allows for the access and reassembly of data in
a database without the need to entirely reorganize the
database. \cite{searchsql} A simple example of the flexibility of a
relational database is the ability to add new data categories without
having to modify existing applications. \cite{searchsql} The
categories in the table correspond to the columns in the table while
each row of a table defines a separate entry into that
table. \cite{ibm} In more abstract terms, the columns of a table are
referred to as attributes and the rows or entries in a table are
referred to as tuples. This framework provides the structure by which
data can be logically organized, thus allowing for easy access and
analysis of the data. The data in the various tables of the database
can be analyzed and filtered based on various sets of criteria and the
results presented in a view to an end user. \cite{searchsql}

A simple top down overview of the structure of, say, a database
managed with MySQL is database>table>entry. A database consists of a
table or tables. Every table contains columns that attempt to define
the type of data that is to be stored in the column. Tables are in
turn made up of rows which correspond to separate entries in the
table. A potentially useful analogy is to think of a database as a
warehouse. The warehouse can be partitioned into different
sections. The different sections of the warehouse are like the tables
in a database. Each section of the warehouse is responsible for
storing a particular type of container. The different containers in
the sections of the warehouse are analogous to the individual entries
in a table. These containers are themselves partitioned further so
that corresponding sections of different containers contain the same
type or category of stuff (data). The various sections within the
containers are similar to the different columns in a table. While the
utility of this analogy may be limited, it may prove useful in
providing a basic idea of the different parts that allow the
structured storage of data in a database.

\section{RDBMS}
Relational Database Management Systems (RDBMS) are programs that allow
for the creation, updating and administration of relational
databases. \cite{searchsql} Many commercial RDBMS softwares use
Structured Query Language to access and interact with the relational
databases they manage. \cite{techrdbms} SQL is a standardized
programming language used to perform tasks in relational databases
such as creating databases and tables, adding and deleting entries,
and querying databases and tables. \cite{techsql} Databases that
utilize the SQL language are colloquially referred to as SQL
databases. \cite{techsql} According to \cite{techsql}, "an official
SQL standard was adopted by the American National Standards Institute
(ANSI) in 1986 and then by the International Organization for
Standardization, known as ISO, in 1987." Although a standard SQL
syntax is defined, many RDBMSs have added their own extensions to the
language which, in many cases, can only be used with their own
systems. \cite{sqlcourse}

\section{MySQL Use Cases}
MySQL has been used by many different organizations to solve many
different problems. \cite{mysqlcase}. The number of organizations that
are listed in \cite{mysqlcase}, and the wide array of problems that
they use MySQL to solve, demonstrates the flexibility and utility of
relational databases managed by MySQL. Some of the applications that
MySQL has been applied to range from science to marketing to
health-care. \cite{mysqlcase} These applications have a wide ranging
impact on those that utilize their services. \newline
\newline

\subsection{CERN}
MySQL has been employed at the European Organization for Nuclear
Research (CERN). \cite{mysqlcern} Scientists at CERN use sensitive and
complex instruments to study the fundamental building blocks and
origin of our universe. \cite{mysqlcern} The IT department at CERN has
offered a MySQL Database-as-a-Service to its employees and the
scientists associated with its various projects to aid in the
management of the data it collects. \cite{mysqlcern} The management of
the database itself is handled by CERN's IT staff but is made
available to those with responsibilities and interests in analyzing
the data collected. \cite{primarynumbers}

A specific project at CERN that utilizes MySQL is the ATLAS
project. \cite{primarynumbers} ATLAS uses MySQL to manage its Primary
Numbers database. \cite{primarynumbers} These Primary Numbers are
parameters that describe the "detector geometry and digitization in
simulations, as well as certain reconstruction parameters, including
the identifier maps of the detector elements." \cite{primarynumbers}
An example of the type of data stored in the Primary Numbers database
is are maps of magnetic fields associated with instrumentation used in
the experiment. \cite{primarynumbers} One of the specific benefits
that MySQL offers to the group at CERN, with relation to managing the
Primary Numbers database with MySQL, is that the database can be used
by a researcher although they are not currently connected to the
central database. \cite{primarynumbers} Once they are connected to the
database again, the database will be updated. \cite{primarynumbers}
Other features that have been found useful include the transfer of
data in binary form and MySQL's certificate authorization
technology. \cite{primarynumbers}

\subsection{boo-box}
boo-box is an advertising network that primarily focuses on the South
American market. \cite{boo} boo-box facilitates the coupling of
marketing platforms with marketers to display approximately one
billion advertisements per month. \cite{boo} boo-box connects
marketers with marketing space on "web sites, blogs and social media
properties." \cite{boo} MySQL is used by boo-box to log user activity
such as what pages they view and their click-through rates. \cite{boo}
Analyzing this data helps boo-box to accurately report and direct ads
in marketing campaigns. \cite{boo}

MySQL has offered boo-box low latency query performance which allows
them to place targeted advertisements in under 250
milliseconds. \cite{boo} Ruby and Python interact with boo-box's MySQL
database to offer other services in their advertising
platform. \cite{boo} boo-box is able to capture 2 TB of web logs and
process 22 billion rows with MySQL. \cite{boo} boo-box is able to use
MySQL to store over 8 TB of data and manage 1 TB in a Statistics
database. \cite{boo}

\subsection{Health-care}
In the health-care industry, MySQL can be used to manage data related
to scheduling, billing, prescriptions, and Electronic Medical
Records. \cite{health} A couple of the benefits of using MySQL in a
health-care application include scalability and security options (with
MySQL Enterprise Audit) allowing for compliance with applicable
regulations. \cite{health} Waiting Room Solutions (WRS) is a web-based
service for physicians' offices that uses MySQL. \cite{health} They
offer solutions for the management of "electronic medical and health
records, billing, scheduling, electronic prescriptions, [and] online
patient registration" among many other services. \cite{health}

\section{Education}
MySQL has been available for download for free since
2000. \cite{mysqlhist} As such, it should come as no surprise that
there are plenty of educational resources available free of charge to
those who wish to learn how to use MySQL. \cite{mysql50} provides a
list of 50 different sources that offer lessons about various aspects
of using MySQL. w3shools.com offers free interactive lessons that
cover the basics of the SQL language, which of course is an essential
skill if one would like to work with and understand
MySQL. \cite{w3mysqlintro} An educational resource that focuses more
on MySQL can be found at www.mysqltutorial.org. \cite{mysql50} The
MySQL Tutorial website offers many lessons in using MySQL; from a
Basic MySQL Tutorial to MySQL Administration to programming interfaces
such as a Python MySQL Tutorial, this site is a near exhaustive
resource to learn all things related to MySQL. \cite{mysqltutorial}

\section{Alternatives}
MySQL is not the only RDBMS. Different RDBMSs have advantages and
disadvantages in any given situation/application. \cite{mysqlcomp} Two
other RDBMSs that are worth comparing to MySQL are SQLite and
PostgreSQL. \cite{mysqlcomp} These three RDBMSs provide end users with
a wide array of options, each possibly more suitable than the next in
different situations.

\subsection{SQLite}
SQLite is an embedded RDBMS. \cite{mysqlcomp} It is a file-based
database that contains a many tools that enable it to deal with a wide
array of data types. \cite{mysqlcomp} SQLite is faster than server
relational databases. \cite{mysqlcomp} One of the main disadvantages
of SQLite is that it doesn't offer user management. \cite{mysqlcomp}
Depending on the intended use, this can be a real concern if the
application requires multiple users to access the
database. \cite{mysqlcomp}

\subsection{PostgreSQL}
PostgreSQL is another open-source RDBMS that is compliant with
applicable SQL standards. \cite{mysqlcomp} PostgreSQL is extensible,
meaning that functions can be stored as procedures. \cite{mysqlcomp}
In addition to being a a relational database management system,
PostreSQL is also objective and thus supports
nesting. \cite{mysqlcomp} One disadvantage of PostgreSQL is that it
does not offer fast read operations. \cite{mysqlcomp}

% Bibliography
\bibliography{references} 

\end{document}


\documentclass[9pt,twocolumn,twoside]{styles/osajnl}
\usepackage{fancyvrb}
\journal{i524} 

\title{\LaTeX\ vCloud and vSphere}

\author[1] {Michael Smith}


\affil[1]{School of Informatics and Computing, Bloomington, IN 47408, U.S.A.}


\affil[*]{Corresponding authors: mls35@iu.edu}

\dates{Paper 1, \today}

\ociscodes{vCloud, vSphere, Iaas, I524}

% replace this with your url in github/gitlab
\doi{\url{https://github.com/cloudmesh/classes/blob/master/docs/source/format/report/report.pdf}}


\begin{abstract}
vSphere and vCloud are categorized as infractructure as a
service(Iaas) offering.  vSphere was developed by VMware and is a
cloud computing virtualization platform.\cite{www-vmware} vSphere is
not one piece of software but a suite of tools that contains software
such as vCenter, ESXi, vSphere client and a number of other
technologies.  Similarly, vCloud is also a suite of applications but
for establishing an infrastructure for a private
cloud.\cite{www-mustbegeek} The suite includes the vsphere suite,
but also contains site recovery management for disaster recovery, site
networking and security.  These services as well as Iaas in general
are discussed.  Advantages and disadvantages of utilizing such as
service are presented followed by an analysis of vSphere and vCloud.
The big data extensions available for these services are examined.
\end{abstract}

\setboolean{displaycopyright}{true}

\begin{document}

\maketitle

\section{Cloud Computing}

With the rise of high speed internet, connectivity amongst devices
locally made cloud computing a practical alternative to the standard
local solution.  Cloud computing is defined as the utilization of
remote servers to provide the computing power and data management
instead of utilizing a local computer.  During the late 1990's,
companies started to see the potential benefits of moving to the
cloud, one of the first companies to do so was salesforce.com how
delivered applications to end users over the internet. \cite{www-eci}
Today there are many companies in the business of cloud computing.
The majority of business coming from Microsoft, Google, and Amazon.
There are subdivisions of cloud computing such as software as
service(Saas), platform as a service(Paas), and infrastructure as a
service(Iaas).  Saas is software that runs remotely from a service
provider and is utilized over the internet, some examples that call
into this category are google docs, gmail, office, and office 365.
Paas are web and database services enabling the user to develop and
run applications in the cloud, some examples are google app engine,
microsoft azure, and amazon web services.  Iaas provides the user
virtualized computing resources over the internet, amazon web
services, Microsoft azure, and VMware.  It is important to note that
many providers offer more than just one type of cloud computing
option. \cite{www-rackspace}



\section{Advantages of Iaas}


Iaas is a business model in where a provider will host all components
of infrastructure including hardware, software, servers, data storage
for their clients.  The cost to the client is at a rate of usage can
range from different rates such as per hour, week or month.  While it
might seem costly to pay for such as service, utilization of Iaas
provide several advantages for organizations to use such as service.
Management does not have the upfront costs their own IT infrastructure
as well as maintaining hardware or replacing equipment that is
obsolete.  The responsibility of ensuring the network is and up and
running through having a qualified team of IT staff is also mitigated.
The cost of Iaas is metered thus companies will only pay for what they
need. \cite{www-statetech} Depending on the application of an Iaas, there
may be instances where a company may need to quickly scale up their
usage.  A great example is when a website produces an unexpected
influx of visits in short period.  If the IT infrastructure is
maintained in house, it might not be able to quickly scale up to the
demand.  This would be detrimental to the business due to a slow
online experience of the customer or possibly a website that
crashes. \cite{www-linkedin} It is both costly and impractical to
staff IT personnel to monitor demands twenty four hours a day seven
days a week.  The Iaas has the capacity to scale with depends and
solutions are in place to help alleviate instances where a sudden
increase in IT infrastructure is required.  Disaster recovery is a
major worry for companies, implementation of a safe backup solution
can be both costly and difficult to execute.  Iaas offer disaster
recovery solutions, as long as an internet connection is available the
same environment can be accessed from leading to little to no
downtime.  With offsite storage of data via Iaas any form of data loss
is potential is limited to very little or none. \cite{www-statetech}

\section{Disadvantages of Iaas}

Iaas is not a perfect solution, it does come with potential problems.
Security is a major concern, becoming fully reliant on a service
provider for IT solutions means their outages become a major problem
for the client’s business.  Sensitive data that is being protected in
the cloud always has the risk of possibly being stolen.  As these Iaas
services grow with a list of companies they support, this can lead
them to become appealing targets for possible hackers due to the
degree of damage that can be done.  A common threat to all websites
that Iaas are not immune to are called distributed denial of service
(DDos) attacks.  This is a type of cyber-attack where the goal is to
make a network unavailable for normal users by disrupting service of a
host on the internet. \cite{www-wikipedia} The typical method to
disrupt service by flooding the bandwidth of a host with plethora of
requests through a botnet which is definied as a network of computers
programmed to receive commands without the owners knowledge.  These
attacks have grown to such an extent that the department of homeland
security has initiated funding new research leading to the prevention
of DDos. \cite{www-ciodive} From a business perspective, this
vulnerability can have disastrous effects to the users of Iaas, risks
should be assessed and possible backup plans in place.

\section{vCloud and vSphere}

vCloud falls into the cloud computing subcategory Iaas.  It is a suite
of multiple products that include vSphere, vCloud Director, vCloud,
connector, vCloud networking and security, vCloud networking and
security, vCenter site recovery and manager, vCenter operations
management suite, vFabric application director, and vCloud automation
center.  vSphere is responsible for the physical hardware resource
management and allocation of virualization across alarge group of
infrastructure such as CPUs, data storage and networking.

vSphere utilizes ESXi which is an enterprise class type 1 hypervisor.  A
hypervisor also known as a virtual machine manager is defined as “a
hardware virualization technique that allows multiple guest operating
systems (OS) to run on a single host system at the same time.  The
guest OS shares the hardware of the host computer, such that each OS
appears to have its own processor, memory and other hardware
resources.” \cite{www-technopedia} The type 1 refers to a type of
hypervisor that can run directly on the hosts pc to control its
resources.

vCloud director is a tool for overall cloud management it
helps the user build hybrid clouds through pooling resources into data
centers.  This product helps empower exisiting IT within an
organization the tools necessary to expand their infrastructure into
the cloud. vCloud connector creates a single user interface that as a
bridge between private and public clouds, this simplifies management
by enabling the user to transfer workloads under a single hybrid cloud
umbrella.

vCloud networking and security provides capabilities to
protect virtual machines.  A firewall is applied either encompassing a
virtual datacenter or at the network interface.  VPN or virtual
private network is utilized to ensure safety for extensions of the
virtual data center, as well as secure sockets layer (SSL) VPN which
is an industry standard for security compliance. \cite{www-vmware3}
With regards to data security, a feature included will scan file
servers for sensitive data such as credit card or social security
numbers and ensure proper measures are place for protection of such
critical data.

Cloud Disaster recovery is defined as “a backup and
restore strategy that involves storing and maintaining copies of
electronic records in a cloud computing environment”. \cite{www-cloud}
This also addressed in the feature vCenter site recovery manager.  It
is an automated solution that will recovery from downtimes in a timely
manner.  This is done so by using replication technology to migrate
virtual machines to a different site.  Users are able to test the
migration process in order to safely address any potential migration
issues.

\section{Licensing}

Depending on the demand, the cost of vCloud can be quite expensive.
It is important to note that there are currently Iaas offerings that
are free but with limit usage.  Amazon web services offers a free tier
consisting of certain limits such as 1 million requests per month on
aws lambda, 25 gb of storage through dynamoDB, 100 million free events
per month on amazon mobile analytics and many other
limits. \cite{www-aws} Other services generally offer limited time
services followed by a pay requirement.

\section{vSphere Big Data Extensions}

Within the vSphere suite is a feature that can support big data and
Apache Hadoop workloads.  A set of tools are available for the user to
deploy and run Haddop within the virtual infrastructure.  The
following distributions of Hadoop are supported: apache Hadoop,
cloudera, pivotal, hortonworks, and mapR.  Customization options such
deploying a specific version of Hadoop or even multiple types of
Hadoop are supported.  In order to automate the management of Hadoop,
VMware initiated project Serengeti.  It can quickly deploy a Hadoop
cluster into vSphere, it will protect master node by automatically
starting a new virtual machine if there is a suspected failure.  An
important note is that graphical user interface of big data extensions
is only supported on the web client 5.1 or later, if big data
extensions is installed on vSphere 5.0, all abilities of the
administrator can only be performed within the
command-line. \cite{www-vmware2}

\section{VCloud API}

The vcloud application program interface (API) clients and the
director communicate via HTTP through an XML exchange. “You use HTTP
GET requests to retrieve the current representation of an object, HTTP
POST and PUT requests to create or modifty an object, and HTTP DELETE
requests to delete an object.” \cite{www-api}


\section{Conclusion}

VMware is one of the oldest companies that have been involved in the
virtualization market.  Their service provides a suite of tools that
assist companies who want to utilize IT virtual infrastructure.  The
Iaas can help businesses in a lot of areas that would be difficult in
a private IT environment such as scalablity, cost, disaster recovery
and backup.  There are drawbacks to Iaas such as risk of downtime from
the IT service provider and security risks of sensitive dat, however
the vCloud suite has features that can address these drawbacks with
regards to security.  Additionally, vcloud supports big data
extensions including support for various versions of hadoop and have
initiatives such as project serengeti which will automate its
management. Vcloud is a potential option for clients interested in
extending their IT infrastructure into the cloud.

% Bibliography

\bibliography{references}

\end{document}

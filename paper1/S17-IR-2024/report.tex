\documentclass[9pt,twocolumn,twoside]{../../styles/osajnl}
\usepackage{fancyvrb}
\journal{i524} 

\title{LDAP}

\author[1]{Ronak Parekh}
\author[2]{Gregor von Laszewski}

\affil[1]{School of Informatics and Computing, Bloomington, IN 47408, U.S.A.}

\affil[*]{Corresponding authors: parekhr@indiana.edu}

\dates{project-000, \today}

\ociscodes{LDAP, directory, authentication, X.500, commands, design}

% replace this with your url in github/gitlab
\doi{\url{https://github.com/ronak1182/sp17-i524/tree/master/paper1/S17-IR-2024/report.pdf}}


\begin{abstract}
  LDAP is a “lightweight” (smaller amount of code) version of
  Directory Access Protocol (DAP), which is part of X.500, a standard
  for directory services in a network. In a network, a directory tells
  you where in the network something is located. On TCP/IP networks
  (including the Internet), the domain name system (DNS) is the
  directory system used to relate the domain name to a specific
  network address (a unique location on the network).LDAP allows you
  to search for an individual without knowing where they’re located
  (although additional information will help with the search).An LDAP
  directory can be distributed among many servers. Each server can
  have a replicated version of the total directory that is
  synchronized periodically. \newline
\end{abstract}

\setboolean{displaycopyright}{true}

\begin{document}

\maketitle

\section{Introduction}

LDAP (Lightweight Directory Access Protocol) is an open,
vendor-neutral, industry standard application protocol for accessing
and maintaining distributed directory information services over an
Internet Protocol network.\cite{www-ldap-wikipedia} LDAP is based on the
client/server model of distributed computing. It has evolved as a
lightweight protocol for accessing information in X.500 directory
services.\cite{ldap-ibm-book} LDAP is specified in a series of
Internet Engineering Task Force(IETF) Standard Track publications
called Request for Comments (RFC's), using the description language
ASN.1. LDAP is based on a simpler subset of the standards contained
within the X.500 standard.

\section{Background}
LDAP was a result of the X.500 series of International
Telecommunication Union (ITU) recommendations.  X.500 is a set of
recommendations about directories. \cite{ldap-ibm-book} Because of this
relationship, the structure of X.500 and LDAP is similar. LDAP
directory implementations are often X.500 compliant and gateways
between the two directories are plentiful. LDAP is defined by a set of
published Internet standards, commonly referenced by their Request For
Comment (RFC) number. The main reason for the emergence of LDAP can be
attributed to the fact that X.500 was too tied to the OSI (Open
Systems Interconnection) protocols, making it out of favour for the
TCP-dominated world that was emerging. LDAP and Domain Naming System
solved problems in a simpler way and thus, LDAP started becoming
widely used. \cite{ldap-ibm-book}

\section{Why LDAP over X.500?}
The success of LDAP has been largely due to the characteristics that
makes it simpler to implement and use as compared to X.500. LDAP runs
over TCP/IP rather than the OSI protocol stack. The availability of
TCP/IP is more and it consumes less resources. The functional model of
LDAP is simpler i.e it removes less frequently used, duplicate and
esoteric features, making it easier to implement and understand. LDAP
uses strings to represent data rather than Abstract Syntax Notation
One (ASN.1) which is considered much more
complicated.\cite{ldap-ibm-book}

\section{LDAP Architecture Overview}
LDAP defines the content of messages exchanged between an LDAP client
and an LDAP server. There are some operations such as search, modify,
delete which are specified by the messsages requested by the client or
responses from the server.  The messages also describe the format of
the data it carries \cite{www-directory-organization}. The messages
are carried over a TCP/IP protocol and thus, there are operations to
establish and disconnect a session between the client and the server.
The factors to be considered while designing the LDAP directory are
the logical model which is defined by the messages and data types, the
organization structure of the directory, the possible operations to be
perfomed on the directory and the security of the information carried
in the messages.\cite{www-ldap-wikipedia}

The general interaction between an LDAP client and LDAP server is as
follows:

The first step is known as Binding step. In this step, the client
establishes a session with the LDAP server. The client specifies the
host name or the IP address and the TCP/IP port number where the LDAP
server is listening. The client can provide a user name and a password
to properly authenticate with the server. Data encryption can also be
used while establishing a session to inmprove security.  The client
then performs operations on directory data. LDAP offers both read and
update capabilities. Thus, the directory information can be managed as
well as queried. Searching is a common LDAP operation where user can
specify what part of the directory to search and what information to
return.  When the client is finished making requests, it closes the
session with the server which is known as unbinding.  The LDAP
directory stores and organizes data structures known as entries.  A
directory entry usually descibes an object such as a person or a
server and so on. Each entry has a distinguished name (DN) that
uniquely identifies it. The DN consists of a sequence of parts called
relative distinguished names (RDNs). The entries can be arranged into
a hierarchical tree-like structure based on their distinguised
names. The tree of directory entries is called the Directory
Information Tree. \cite{ldap-ibm-book}

Each entry contains one or more attributes that describe the
entry. Each attribute has a type and a value. A directory entry
describes some object which in general is called as a template.
The following operations are defined by LDAP for accessing and modifying
directory entries:
\begin{itemize}
\item Adding an entry
\item Deleting an entry
\item Modifying an entry
\item Comparing an entry
\item Moving an entry
\end{itemize}

A client starts an LDAP session by connecting to an LDAP server,
called a Directory System Agen (DSA), by default on TCP and UDP port
389, or on port 636 for LDAPS. The client sends an operation request
to the server, and the server sends responses in return. The client
does not need to wait for a response before sending the next request,
and the server may send the responses in any order. The client may request
the following operations: StartTLS, Bind, Search, Compare, Add entry,
Delete Entry, Modify Entry, Abandon, Extended Operation, Unbind. The
server may send Unsolicited Notifications" that are not responses to any
request.

\section{Designing and Maintaining LDAP directory}
The designing of an LDAP directory are distributed into four phases
\cite{www-ldap-design}
\begin{itemize}
\item First Phase: The first phase is defining the directory
  content. It has two components: First component is to define the
  directory requirements which is to carefully analyse the main
  purpose of the directory and the consideration to arrive to a
  holistic approach for the directory plan.  The second component is
  the designing of data which is to understand the source and nature
  of the data. The scope of the data within the directory is decided
  and its integration with external data is
  planned. \cite{www-ldap-design}

\item Second Phase: This phase is also divided into two
  components. First is designing the schema and determining the format
  in which the data is to be stored. The second component is desinging
  the namespace and determining the hierarchical structure of the
  directory.

\item Third Phase: This phase involves securing the directory
  entries. The privacy and security of the data in the directory is
  the main area of focus.  The applications using the directory should
  also be secured and their security is also considered in this phase.

\item Fourth Phase: This phase involves in designing the underlying
  network infrastructure and designing the server. This phase involves
  the topology design which helps to determine the number of servers
  and the location of their directory services. It also considers
  about the distribution of data amongst the servers. Replication can
  also be considered in this phase, which enables multiple copies of
  the data to be deployed. \cite{www-ldap-design}
  
\end{itemize}

\section{LDAP Command-Line Tools}
LDAP protocol operations are divided into three categories:
Authentication, Interrogation, and Update and Control.  The LDAP C-API
provides a number of simple command-line tools that together covers
all three categories \cite{www-ldap-commands}.
\begin{itemize}
\item ldapbind: It authenticates to a directory server. It can also be
  used to check whether a particular server is running or not.
\item ldapsearch: It searches for specific entries in a directory. It
  opens a connection to a directory, authenticates the user performing
  the operation, searches for the specified entry, and prints the
  result of the search operation.
\item ldapadd: It adds an entry to the directory. It opens a
  connection to the directory and authenticates the user and opens the
  LDIF file supplied as an argument and adds each entry in the file in
  succession.
\item ldapdelete: It removes the leaf entries from a directory. It
  deletes the specified entry by opening a connetion and
  authenticating the user.
\item ldapmodify: It modifies the existing entries by opening a
  connection to the directory and opening the LDIF file supplied and
  then modifies the entry.
\item ldapmoddn: It changes the RDN of an entry. It moves an entry or
  subtree to another location in the directory.
\end{itemize}

\section{Future of LDAP}
The future of LDAP lies in refinements to LDAPv3. The most recent
improvements added include upgrades to managements GUIs that allow
easier modification of users and their attributes. The greatest
challenge is one shared with any other directory service which
includes Active Directory. LDAP's ability to adapt to the changes in
delivery of identity and access management which could be possible
through new types of authentication such as biometrics or through
Software as a Service (SaaS) models. The key features to LDAP's future
are its ability to be flexible, scalable and adaptable with new
technologies. \cite{ldap-ibm-book}

\section{Conclusion}
By reviewing the features and implementation of LDAP, we can conclude
that LDAP is lightweight and a Standard wire protocol. LDAP has
emerged as a mature protocol for accessing directories. Its servers
provide a number of security features such as automatically encoding
passwords with one-way digests, its support for extensible
authentication via the SASL framework. It includes general purpose
data storage which can hold information about users and
groups. Alongwith a variety of features LDAP supports high
availability, disaster recovery and logging options to boost its
usability. Thus, making LDAP extremely effective.



% Bibliography

\bibliography{references}
 
\end{document}

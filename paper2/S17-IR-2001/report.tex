\documentclass[9pt,twocolumn,twoside]{../../styles/osajnl}
\usepackage{fancyvrb}
\journal{i524} 



\dates{project-000, \today}

\title{HUBzero: A Platform For Scientific Collaboration}

\author[1,*]{Niteesh Kumar Akurati}

\affil[1]{School of Informatics and Computing, Bloomington, IN 47408, U.S.A.}

\affil[*]{Corresponding authors: akuratin@indiana.edu}

\dates{paper2, \today}

\ociscodes{cyberinfrastructure, BigData, Simulation Tools, Collaboration }

% replace this with your url in github/gitlab
\doi{\url{https://github.com/cloudmesh/sp17-i524/blob/master/paper2/S17-IR-2001/report.pdf}}


\begin{abstract}
HUBzero is a cyberinfrastructure in a box that is used to create
dynamic websites for educational activities as well as scientific
research. HUBzero provides an platform where researchers can publish
and share their research software and related material for
educational purposes on the web. Other researchers will be able to
access the tools as well as material using a Web browser and can also
launch simulation runs on the national Grid infrastructure without
having to download any code.\newline
\end{abstract}

\setboolean{displaycopyright}{true}

\begin{document}

\maketitle

\section{Introduction}
HUBzero is a platform created to support nanoHUB.org, an online
community for the Network for Computational Nanatechnology(NCN), which
is funded by the U.S.National Science Foundation since 2002 to connect
the theorists across the academia who develop simulation tools with
the educators and the experimentalists who might use them. HUBzero has
been expanded to more than 30hubs since 2002 and the hubs are growing
every year into various fields ranging from development of assistive
technology to volcanlogy\cite{mclennan2010hubzero}.

''HUBzero is now supported by a consortium including Purdue, Indiana,
Clemson and Wisconsin. Researchers at Rice, the State University of
New York system, the University of Connecticut and Notre Dame use
hubs. Purdue offers a hub-building and -hosting service and the
consortium also supports an open source release, allowing people to
build and host their own''\cite{hpcwireacrticle2011}.

A hub web 2.0 functionality with an unique middleware, thereby
providing a platform that is more powerful than an ordinary website.
In a hub users will able to create, publish, share information,
network and also have access to interactive visualization tools. The
users of the hub ultimately collaborate to develop the educational
community and make it a powerful resource. A hub can direct jobs to
national resources such as TeraGrid, Purdue's DiaGrid as well as other
cloud systems\cite{hpcwirearticle2011}.


\section{Middleware}
HUBzero mix of social networking and simulation is only possible with
the help of unique middleware. HUBzero's middleware hosts the live
session tools and makes it easy to connect the cloud computing
infrastructure and supercomputing clusters to solve the large
computational problems. The simulation tools runs on clusters of
executions hosts hosted near the Web server and will be accessed by
user's browser via VNC(Virtual Network Connection)\cite{mclennan2010hubzero}.

Each user has a home directory which is unique to that particular user
with access control, quota limitations and conventional ownership
privileges. Each tool runs on a lightweight restricted virtual
environment, that controls access to the networking, file systems and
server processes. Tools are run with the privileges of a particular
user\cite{mclennan2010hubzero}.

Tools running on the hub will have a XII Window System environment.
Each container runs a special X server which also acts as a VNC server
is used to create the graphical session. The middleware of HUBzero
controls the tool container's network operations. The middleware also
monitors the start time as well as the duration of each
connection. The connections that are not being used for a considerable
period after starting is terminated and marked as
inactive\cite{hubzeropurduewebpage}.

\section{Features}
HUBzero unique blend of social networking and simulation power made it
popular among research and scientific communities. HUBzero's platforms
are sophisticated because of the various features it offers. HUBzero
sites provides a lot of features for collaboration, development and
deployment of tools, interactive user interfaces for simulation tools,
online community groups, Wikis, Blogs, Provision for Feedback
Mechanisms, Metrics for Usage, content management resources and
everything that would be needed for a collaborative platform

\subsection{Development and Deployment of Tools}
Each hub comes with a companion site for developing source code based
on open source package called Trac for project management. Uploading a
tool is a little complicated. Tools must be uploaded, compiled,
tested, fixed, compiled again, and tested again many times before
being published.Each tool will have its own project area within the
site, with a repository for source code control, a ticketing system
for bug tracking as well as a wiki area for project documentation.

A team member from development get access to workspace which is a
Linux desktop running in a secured execution environment, accessed via
a Web browser. The tools that are available on the hub dont come from
the core development team rather they are developed by various
collaborators across the world. Developers can use HUBzero's rapture
toolkit to create a GUI with little effort\cite{hubzerofeatures}.

\subsection{Online Presentations}
HUBzero facilitates Online Presentations along with the tools. The
PowerPoint slides are combined with voice and animation. The online
presentations provide user a standard seminar experience. HUBzero uses
MacroMedia Breeze to deliver the online presentations in a compact
format using Flash, which is present on 98 percent of the world
desktops. The online presentations provided by HUBzero can also be
distributed as podcasts which be accessed by the users on-the-go via
their personal device assistants\cite{hubzerofeatures}.

\subsection{Interactive Simulation Tools}
The underlying premise of HUBzero is its signature servicce to deliver
live graphical simulation tools that are interactive and can be
accessed using a normal web browser. The tools in a hub are
interactive; you can zoom a molecule, analyze 3D volume interactively
and need not even wait for the webpage to be refreshed. A user can
visualize results in real-time and need for wait for processing. One
can deploy new tools without having to write any special code for
deploying on the web\cite{hubzerofeatures}.

HUBzero's Rappture Toolkit helps to create GUI with little effort. The
HUBzero infrastructure provides tool execution as well as delivery
mechanism based on Virtual Network Computing(VNC). Using the
architecture provided by HUBzero, the first ever developed hub,
nanoHUB has brought about 50 simulation tools live in span of 2 years.

\subsection{Collaboration and Support}
HUBzero is a popular because it not only provides simulation tools,
wikis etc., but also provides sophisticated mechanism for colleagues
to collaborate among themselves and work together. HUBzero middleware
hosts tool sessions, where a single session can be shared with among
many number of people. Each session provides a textbox entry beneath
it to allow the users to enter the colleagues names they wish to
collaborate with.  A group of people can see the same session at the
same time and discuss ideas over the phone or instant messaging\cite{hubzerofeatures}.

HUBzero also provides community forum modeled after Amazon.com's
Askville where users can post questions. Users can also tag the posts
with the tool names and concepts so that users can find questions
matching their interests and post answers accordingly. Users also can
file trouble reports if something goes wrong which are sent directly
to tool development team. Developers view the help requesrs as tickets
and investigate and resolve the issues. Users also can have access to
wishlist where they can request new features for a particular
tool. The developers cam determine a wish's relative priority and
judges which are important and requires little effort are ordered
accordingly. There is points and bounty system which enhances
involvement of users in the hub by providing premium access to the
resources based on the levels of bounties and points

\subsection{Content Tagging, Wikis abd Blogs}
The resources available on a hub are categorized by a series of
tags. Each tag has an associated page on the hub where its resources
are listed and meaning is defined. Tags can be defined by anyone like
the contributor, users of the page or the hub adminstrator.

Hubs support topic pages. Each topic is created using a standard wiki
syntax by a specified list of authors. Users can comment on the
content of the wiki or even suggest changes. The original moderators
of the topic are notified about the suggestions which they will
consider if apt. The page can also be given a ownership. Topic pages
act as lightweight articles that help to describe various resources on
the hub\cite{hubzerofeatures}.

\section{Applications}
HUBzero platform has been very popular in sciece and engineering
fields. Many hubs have come into existence in various fields ranging
from cancer to nanotechnology. More than 30 hubs have been spawned
since HUBzero platform had come into existence. All these hubs serve
more than 8,50,000 visitors from 172 countries worldwide during year
2012 alone.

\subsection{NanoHUB}
nanoHUB is the reason behind the HUBzero platform. HUBzero was
initially created to support nanoHUB.org, an online community for the
Network for Computational Nanotechnology, which is funded by US
National Science Foundation in 2002 to connect theorists who develop
simulation tools with the experimenters and engineers.

nanoHUB is an online portal for nanotechnology where researchers,
students and instructors collaborate to share scientific tools,
simulation tools, educational material, research etc., It uses
cyberinfrastructure to provide access to this tools and also the
instructional materials. The users can run experiments, download
lectures as well as review research\cite{klimeck2008nanohub}.

\subsection{Big Data Analysis in Social Science Using HUBzero}
Datasets from social media are infact large and can grow beyond a
individual or institution analytical capability of common software
tools. Institutions such as Non-STEM or undergraduate schools lack the
compute infrastructure and personnel needed to allows researchers,
students, instructors to create and analyze large datasets.

The social science students are expected to be competent
users. Therefore, in order to provide infrastructure necessary to
expose undergraduates social science students to data intensive
computing, State University of New York (SUNY) Oneonta teamed with
University at Buffalo's Center for Computational Research (CCR) to
establish a collaborative virtual community focusing on data intensive
computing education\cite{BigDataApplication}.

\section{License}
HUBzero has been released as Open Source under the LGPL-3.0
license. HUBzero is community code unlike the GNU General Public
License it doesn't ``infect'' your code. Under HUBzero's license, one
can treat HUBzero as a “library,” create your own unique components
within our framework, and license your derivative works any way you
like.

\section{Conclusion}
HUBzero has been a unique platform with social networking and
simulation power has been able to resonate science and engineering
communities. As hub keeps to grow the capabilities continues to grow,
more tools and related content will be added. A hub lets various
researchers, engineers collaborate to solve larger problems by
connecting a series of models from independent authors.




% Bibliography

\bibliography{references}
 


\newpage

\appendix



\end{document}

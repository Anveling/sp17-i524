\documentclass[9pt,twocolumn,twoside]{../../styles/osajnl}
\usepackage{fancyvrb}
\journal{i524} 

\title{Memcached}

\author[1]{Ronak Parekh}
\author[2]{Gregor von Laszewski}

\affil[1]{School of Informatics and Computing, Bloomington, IN 47408, U.S.A.}

\dates{project-000, \today}

\ociscodes{Cloud, I524}

% replace this with your url in github/gitlab
\doi{\url{https://github.com/cloudmesh/classes/blob/master/docs/source/format/report/report.pdf}}


\begin{abstract}
  Memcached is a free and  open source, high-performance, distributed
  memory object caching system. It allows the system to make better
  use of its memory. It uses data caching technique to speed up
  dynamic database-driven websites. This reduces the number of times
  an external data source is accessed by the application. Memcached
  service is mainly offered through an API. It allows to take memory
  from parts of the system where you have more memory and allocate it
  to those parts of the system where you have less memory.
\end{abstract}

\setboolean{displaycopyright}{true}

\begin{document}

\maketitle

\section{Introduction}
Memcached is a high-performance, distributed memory object caching
system, generic in nature, but originally intended for use in speeding
up dynamic web applications by alleviating database load
\cite{www-memcached}. Memcached allows you to take memory from parts
of your system where you have more than you need and make it
accessible to areas where you have less than you need. It is a free
and open-source software, licensed under the Revised BSD
\cite{www-bsd-wikipedia} license. Memcached runs on Unix-like
operating systems (at least Linux and OS X) and on Microsoft Windows.
It depends on the libevent library.

\section{Comparison of Memcached and Redis}
\begin{itemize}
  \item Server: Memcached server is multi-threaded whereas Redis is
    single threaded.  As a result, Redis \cite{www-redis-wikipedia}
    can't effectively harness multiple cores.
  \item Distributability: Memcached is very easy to distribute since
    it doesn't support any rich data-structures.  Redis, on the other
    hand, is much harder to distribute without changing
    semantics/performance guarantees of some its commands.
  \item Ease of Installation: Comparing ease of Installation Redis is
    much easier. No dependencies required.
  \item Memory Usage: For simple key-value pairs memcached is more
    memory efficient than Redis.  If you use Redis hashes, then Redis
    is more memory efficient.
  \item Persistence: If you are using Memcached then data is lost with
    a restart and rebuilding cache is a costly process.  On the other
    hand, Redis can handle persistent data. By default, Redis syncs
    data to the disk at least every 2 seconds.
  \item Replication: Memcached does not supports replication.  Whereas
    Redis supports master-slave replication. It allows slave Redis
    servers to be exact copies of master servers.Data from any Redis
    server can replicate to any number of slaves
    \cite{www-memcached-comparison}.
  \item Read/Write Speed: Memcached is very good to handle high
    traffic websites. It can read lots of information at a time and
    give you back at a great response time. Redis can also handle high
    traffic on read but also can handle heavy writes as well.
  \item Key Length: Memcached key length has a maximum of 250 bytes,
    whereas Redis key length has a maximum of 2GB.  When advanced data
    structures or disk-backed persistence are important, Redis is used.
\end{itemize}

\section{Architecture of Memcached}
Memcached implements a simple and lightweight key-value interface
where all key-value tuples are stored in and served from DRAM. The
following commands are used by clients to communicate with the
Memcached servers:
\begin{itemize}
  \item SET/ADD/REPLACE(key, value): add a (key, value) object to the
    cache.
  \item GET(key): retrieve the value associated with a key.
  \item DELETE(key): delete a key
\end{itemize}

Internally, a hash table is used to index the key-value
entries. Memcached uses a hash table to index the key-value
entries. These entries are also in a linked list sorted by their most
recent access time.  The least recently used (LRU) entry is evicted
and replaced by a newly inserted entry when the cache is full
\cite{www-memcached-blog}.


Hash Table: To lookup keys quickly, the location of each key-value
entry is stored in a hash table. Hash collisions are resolved by
chaining: if more than one key maps into the same hash table bucket,
they form a linked list.  Chaining is efficient for inserting or
deleting single keys.  However, lookup may require scanning the entire
chain.


Memory Allocation: Naive memory allocation could result in significant
memory fragmentation.  To address this problem, Memcached uses
slab-based memory allocation. Memory is divided into 1 MB pages, and
each page is further sub-divided into fixed-length chunks. Key-value
objects are stored in an appropriately sized chunk. The size of a chunk,
and thus the number of chunks per page, depends on the particular slab
class. For example, by default the chunk size of slab class 1 is 72
bytes and each page of this class has 14563 chunks; while the chunk
size of slab class 43 is 1 MB and thus there is only 1 chunk spanning
the whole page.  To insert a new key, Memcached looks up the slab
class whose chunk size best fits this key-value object
\cite{www-memcached-blog}. If a vacant chunk is available, it is
assigned to this item; if the search fails, Memcached will execute
cache eviction.


Cache policy In Memcached, each slab class maintains its own objects
in an LRU queue (see Figure 1). Each access to an object causes that
object to move to the head of the queue. Thus, when Memcached needs to
evict an object from the cache, it can find the least recently used
object at the tail. The queue is implemented as a doubly-linked list,
so each object has two pointers.


Threading: Memcached was originally single-threaded.  It uses libevent
for asynchronous network I/O callbacks. Later versions support
multi-threading but use global locks to protect the core data
structures. As a result, operations such as index lookup/update and
cache eviction/update are all serialized \cite{www-memcached-blog}.

\section{Working of Memcached}
In the Memcached system, each item comprises a key, an expiration
time, optional flags, and raw data. When an item is requested,
Memcached checks the expiration time to see if the item is still valid
before returning it to the client. The cache can be seamlessly
integrated with the application by ensuring that the cache is updated
at the same time as the database.  By default, Memcached acts as a
Least Recently Used cache plus expiration timeouts.  If the server
runs out of memory, it looks for expired items to replace. If
additional memory is needed after replacing all the expired items,
Memcached replaces items that have not been requested for a certain
length of time (the expiration timeout period or longer), keeping more
recently requested information in memory.

Working of Memcached Client \cite{www-memcached-java}: Firstly, a
memcached client object is created and it starts calling its method to
get and set cache entries. When an object is added to the cache, the
Memcached client will take that object, serialize it, and send a byte
array to the Memcached server for storage. At that point, the cached
object might be garbage collected from the JVM where the application
is running. When the cached object is needed, the Memcached client's
get() method is called.  The client will take the get request,
serialize it, and send it to the Memcached server. The Memcached
server will use the request to look up the object from the cache. Once
it has the object, it will return the byte array back to the Memcache
client. The client object will then take the byte array and
deserialize it to create the object and return it to the
application. Even if the application runs on more than one server, all
of them can point to the same Memcached server and use it for getting
and setting cache entries. The Memcached client is configured in such
a way that it knows all the available Memcached servers.

\section{Security}
Memcached is mostly used for deployments within a trusted
network. However, sometimes Memcached is deployed in untrusted
networks and enviroments where administrators exercise control over
the clients that are connected. SASL authentication support is
required for Memcached to compile and it requires the binary
protocol. For simplicity, all Memcached operations are treated
equally. Clients with a valid need for access to low-security entries
within the cache gain access to all entries within the cache. If the
cache key can be either predicted, guessed or found by exhaustive
searching, its cache entry may be retrieved
\cite{www-memcached-wikipedia}.

\section{New Features}
Chained items were introduced in 1.4.29. With -o modern items sized
512k or higher are created by chaining 512k chunks together. This made
increasing the max item size (-I) more efficient in many scenarios as
the slab classes no longer have to be stretched to cover the full
space. There was still an efficiency hole for items 512k->5mb or so
where the overhead is too big for the size of the items. This change
fixes it by using chunks from other slab classes in order to "cap" off
chained items. With this change larger items should be more efficient
than the original slab allocator in all cases.Chunked items are only
used with -o modern or explicitly changing -o slab-chunk-max It is not
recommended to set slab-chunk-max to be smaller than 256k at this
time.

\section{Uses of Memcached}
Memcached is used for scaling large websites. Facebook is one of the
largest users of memcached \cite{www-memcached-facebook}. It is used
to alleviate database load. Facebook uses more than 800 servers
supplying over 28 terabytes of memory to our users.  As Facebook's
popularity had skyrocketed, they've run into a number of scaling
issues. This ever increasing demand required Facebook to make
modifications to both our operating system and memcached to achieve
the performance that provided the best possible experience for their
users. There were thousand of computers, each running hundreds of
Apache processes which ultimately led to thousands of TCP connections
open to the memcached processes.  Memcached uses a per-connection
buffer to read and write data out over the network. When hundreds and
thousands of connections were in consideration, memory added up to be
in gigabytes. Memory that could be better used to store user data had
to be considered. Thus, memcached came into consideration while
scaling large websites.

\section{Conclusion}
Memcached is used when scaling large websites is to be done. By
alleviating the database load, it helps making the maximum use of its
caching technique to find hits before the database is queried. This
results in lesser amount of cycles to and from the external data
source or database when it comes to speeding up query results. Memcached
has its main advantage in distributability when it comes to setting up
a distributed cache and gives superior performance when multithreaded
processes are in consideration.

% Bibliography

\bibliography{references}

\end{document}

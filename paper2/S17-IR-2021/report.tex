\documentclass[9pt,twocolumn,twoside]{../../styles/osajnl}
\usepackage{fancyvrb}
\journal{i524} 

\title{Amazon Elastic Beanstalk}

\author[1]{Shree Govind Mishra}

\affil[1]{School of Informatics and Computing, Bloomington, IN 47408, U.S.A.}

\affil[*]{Corresponding authors: shremish@indiana.edu}
\dates{project-000, \today}

\ociscodes{Cloud, I524, Amazon Web Services, AWS Elastic Beanstalk, Load Balancing, Cloud Watch, AWS Management Console, Auto Scaling}

% replace this with your url in github/gitlab
\doi{\url{https://github.com/cloudmesh/sp17-i524/raw/master/paper2/S17-IR-2021/report.pdf}}


\begin{abstract}
Amazon Elastic Beanstalk is a service provided by the Amazon Web
Services which allow developers and engineers to deploy and run
web applications in the cloud, in such a way that these applications
are highly available and scalable. Elastic Beanstalk manages the
deployed application by reducing the management complexities as it
automatically handles the capacity provisioning, load balancing,
scaling, and application health monitoring. Elastic BeanStalk also
provisions one or more AWS Resources such as Amazon EC2 instances when
an application is deployed.
\newline
\end{abstract}

\setboolean{displaycopyright}{true}

\begin{document}

\maketitle

\section{Introduction}

Cloud Computing can be defined as an abstraction of services from
infrastructure (i.e. hardware), platforms, and applications (i.e.
software) by virtualization of resources \cite{elastic-beanstalk}.
The different form of cloud computing services include IaaS, PaaS,
and SaaS which stands for (Infrastructure-as-a-Service),
(Platform-as-a -Service), and (Software-as-a-Service)
respectively. Elastic Beanstalk is a PaaS offered from the AWS that
allows users to create applications and push them over the cloud,
wherein creating an environment where the features and services of AWS
are used such as Amazon EC2, Amazon RDS, etc. to maintain and scale
the application without the need for continuous monitoring.

AWS Elastic Beanstalk is one of the many services provided by the AWS
with its functionality to manage the Infrastructure. Elastic beanstalk
provides a quick deployment and management of the applications on the
Cloud as it automatically handles the details of capacity
provisioning, load balancing, scaling, and application health
monitoring. Elastic beanstalk uses highly reliable and scalable
services \cite{elastic-beanstalk-2}.

\section{Need for AWS Elastic Beanstalk}

Cloud Computing shifts the location of resources to the cloud to
reduce the cost associated with over-provisioning, under-provisioning,
and under-utilization \cite{cloudcomputing}. When more than required
resources are available it is called over-provisioning. When the
resources are not used adequately it is known as
under-utilization. When the resources available are not enough is
called under-provisioning. With Elastic Beanstalk, users can prevent
the under-provisioning, under-utilization and over-overprovisioning of
computing resources, as it keeps a close check on how the workload for
the application is varying with time. The information thus obtained,
helps in automatically scaling the application up and down by
provisioning the adequate required resources to the
application. Elastic Beanstalk also reduces the average response time
for the application as it automatically provides the needed resources
without manual monitoring and decision making. A user uses one or more
of these three scaling techniques to manage and scale the application:

\subsection{Manual Scaling in Cloud Environments}

In traditional applications, scalability is achieved by predicting the
peak loads, then purchasing, setting up and configuring the
infrastructure that could handle this peak load
\cite{ManualScaling}. With Manual Scaling, the resources are
provisioned at the deployment time and the application servers are
added to infrastructure manually, thus there is high latency.

\subsection{Semi-automatic Scaling in Cloud Environments}

In Semi-automatic Scaling the resources are provisioned
dynamically (i.e. at runtime) and  automatically (i.e. without user
intervention) \cite{elastic-beanstalk}. Thus, the mean time until when
the resources are provisioned are short.  However, manual monitoring
of resources are still necessary. Since resources are
provisioned by request, the problem of the unavailability of an
application at peak loads is not completely eliminated.

\subsection{Automatic Scaling in Cloud Environments}

Elastic Beanstalk uses the Automatic Scaling which overcomes the
drawbacks of both the Manual Scaling as well as Semi-automatic Scaling
by allowing the users to closely follow the workload curve of their
application, by provisioning of the resources on demand. The user owns
the choice that the number for resources these applications are using
increases automatically during the time when the demand of resources
are high to handle the peak load and also automatically decreases when
the minimal resources are needed, to minimize the cost so that the
user only pays for what they used. Automatic Scaling also predicts the
peak load that the applications may require in the future and
provisions these required resources in advance, proving the elasticity
of the cloud \cite{elastic-beanstalk}.


\section{Elastic Beanstalk services}

Elastic Beanstalk supports applications developed in Java, PHP, NET,
Node.js, Python, and Ruby, and also in different container types for
each language. A container defines the infrastructure and software
stack to be used for a given environment. When an application is
deployed, Elastic Beanstalk provisions one or moreAWS resources, such
as Amazon EC2 Instances \cite{elastic-beanstalk-book}. The software
stack that runs on the instances depends on the container type, where
two container types are supported by the Elastic Beanstalk Node.js: a
32-bit Amazon Linux Image and a 64-bit Amazon Linux Image. Where each
of them runs the Software stack tailored to the hosted Node.js
application. Amazon Elastic Beanstalk can be interacted using the AWS
Management Console, the AWS Command Line Interface(AWS CLI), or a
high-level CLI designed for Elastic BeanStalk
\cite{elastic-beanstalk-book}.

The following web service and features of AWS are used by the Amazon
Elastic Beanstalk for the Automatic Scaling of Applications:

\subsection{AWS Management Console}

AWS Management Console is a browser-based graphical user
interface(GUI) for Amazon Web Services. It allows users to configure
an automatic scaling mechanism of AWS Elastic Beanstalk as well as
other services of AWS. From the Management console, the user can decide
about how many instances does the application require. When the
application must be scaled up and down \cite{elastic-beanstalk}.

\subsection{Elastic Load Balancing}

Amazon Elastic Beanstalk uses the Elastic Load Balancing service which
enables the load balancer to automatically distributes the incoming
application traffic across all running instances in the auto-scaling
group based on metrics like request count and latency tracked by
Amazon Cloud Watch. If an instance is terminated, the load balancer
will not route requests to this instance anymore. Rather, it will
distribute the requests across the remaining instances
\cite{elastic-loadbalancing}.

\subsection{Auto Scaling}

Auto Scaling automatically launches and terminates instances based on
metrics like CPU and RAM utilization of the application and are
tracked by Amazon CloudWatch and thresholds called triggers. Whenever
a metric crosses a threshold, a trigger is fired to initiate automatic
scaling.  For example, a new instance will be launched and registered
at the load balancer if the average CPU utilization of all running
instances exceeds an upper threshold
    
\subsection{Amazon Cloud Watch}

Amazon CloudWatch enables the application to monitor, manage and
publish various metrics. It also allows configuring alarms based on
the data obtained from the metrics to make operational and business
decisions. Elastic Beanstalk automatically monitors and scales the
application using the CloudWatch.

\section{Amazon Elastic Beanstalk Description}

An Elastic Beanstalk Application is a collection of Elastic Beanstalk
components which include the application versions, environments, and
the environment configurations. Application Version is a deployable
code for the web application and it alson implements the
deployable code via Amazon Simple Storage Service (Amazon S3) which
contains the code. An application may have many different application
versions \cite{elastic-component}.


An environment is a version that is deployed onto AWS resources.  At
the time of environment creation, Elastic Beanstalk provisions the
resources needed to run the application version specified. Where, the
environments include an environment tier, platform, and environment
type. The environment tier chosen determines whether Elastic Beanstalk
provisions resources to support a web application that handles HTTP(S)
requests or web application that handles the background processing
task. AWS resources created for an environment include one elastic
load balancer, an Auto Scaling group, and one or more Amazon EC2
instances \cite{elastic-architecture}.

The software stack that runs on the Amazon EC2 instances is dependent
on the container type. Where a container type defines the
infrastructure topology and software stack to be used for that
environment. For example, an Elastic Beanstalk environment with an
Apache Tomcat container uses the Amazon Linux operating system, Apache
web server, and Apache Tomcat software. In addition, a software
component called the host manager. HM runs on each Amazon EC2 server
instance. The host manager reports metrics, errors and events, and
server instance status, which are available via the AWS Management
Console, APIs, and CLIs \cite{elastic-architecture}. The important
aspects of Elastic Beanstalk such as security, management version
updates, and database and storage are discussed below:

\subsection{Security}

The application on the Elastic Beanstalk cloud is available publically
at myapp.elasticbeanstalk.com for anyone to access. The user can
control what other incoming traffic, such as SSH, is delivered or not
to your application servers by changing the EC2 security group
settings \cite{elastic-faq}.. The IAM (Identity and Access Management)
allows the user to manage users and groups in a centralized manner,
such as the user can control which IAM users have access to AWS
Elastic Beanstalk, and limit permissions to read-only access to
Elastic Beanstalk for operators who should not be able to perform
actions against Elastic Beanstalk resources \cite{elastic-faq}..

\subsection{Management Version Updates}

The user can opt to update the AWS Elastic Beanstalk environment
automatically to the latest version of the underlying platform running
the application during a specified maintenance window
\cite{elastic-faq}..  The managed platform updates use an immutable
deployment mechanism to perform these updates, where the applications
will be available during the maintenance window and consumers will not
be impacted by the update.

\subsection{Database and Storage}

AWS Elastic Beanstalk stores the application files, deployable codes,
and server log files in Amazon S3. If the user is using the AWS
Management Console, the AWS Toolkit for Visual Studio, or AWS Toolkit
for Eclipse, an Amazon S3 bucket will be created in the user's account
for the user, and the files uploaded by the user will be automatically
copied from your local client to Amazon S3 \cite{elastic-faq}. AWS
Elastic Beanstalk does not restrict the user to use any particular
data persistence technology. The user can choose to use Amazon
Relational Database Service (Amazon RDS) or Amazon DynamoDB, or use
Microsoft SQL Server, Oracle, or other relational databases running on
Amazon EC2. The user can configure AWS Elastic Beanstalk environments
to use different databases by specifying the connection information in
the environment configuration. The connection string will be extracted
from the application code and thus Elastic Beanstalk can configure
different databases to work together \cite{elastic-faq}.

\section{Conclusion}

``There is an observation that in many companies the average
utilization of application servers ranges from 5 to 20 percent,
meaning that many resources like CPU and RAM are idle at peak times''
\cite{cloudcomputing-2}. Thus, Amazon Elastic Beanstalk helps to
provide automatic scaling of the application which efficiently
utilizes the expensive computing resources and makes the application
able to manage the peakload at all times \cite{elastic-faq}.

Other platforms like Google App Engine also let users have their
applications to automatically scale both up and down according to
demand but with even more restrictions on how users should develop
their applications \cite{google-appengine}. Wheras, with AWS Elastic
Beanstalk there is more control with the user as the user can manage
all the elements of the infrastructure or choose to go with Automatic
Scaling.

% Bibliography

\bibliography{references}
 


\end{document}

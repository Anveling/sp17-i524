\documentclass[9pt,twocolumn,twoside]{../../styles/osajnl}
\usepackage{fancyvrb}
\journal{i524} 

\title{Amazon Elastic Beanstalk}

\author[1]{Shree Govind Mishra}

\affil[1]{School of Informatics and Computing, Bloomington, IN 47408, U.S.A.}

\affil[*]{Corresponding authors: shremish@indiana.edu}
\dates{project-000, \today}

\ociscodes{Cloud, I524, Amazon Web Services, AWS Elastic Beanstalk, Load Balancing, Cloud Watch, AWS Management Console, Auto Scaling}

% replace this with your url in github/gitlab
\doi{\url{https://github.com/cloudmesh/sp17-i524/raw/master/paper2/S17-IR-2021/report.pdf}}


\begin{abstract}
Amazon Elastic Beanstalk is a service provided by the Amazon Web
Services which allow developers and engineers to easily deploy and run
web applications in the cloud, in such a way that these applications
are highly available and scalable. Elastic Beanstalk manages the
deployed application by reducing the management complexities as it
automatically handles the capacity provisioning, load balancing,
scaling, and application health monitoring. Elastic BeanStalk also
provisions one or more AWS Resources such as Amazon EC2 instances when
an application is deployed.
\newline
\end{abstract}

\setboolean{displaycopyright}{true}

\begin{document}

\maketitle

\TODO{This review document is provided for you to achieve your
  best. We have listed a number of obvious opportunities for
  improvement. When improving it, please keep this copy untouched and
  instead focus on improving report.tex. The review does not include
  all possible improvement suggestions and for each comment you may
  want to check if it applies elsewhere in the document.}

\TODO{Assessment: Major revisions required. You need to focus more on
  Beanstalk and less on AWS in general. A good way to do this is to
  provide use cases. Where you do need to discuss AWS more generally,
  you need to provide better motivation why it's necessary to discuss
  that in a paper on Beanstalk, and you need to make that discussion
  more succinct. Please see below for more detailed comments.}

\TODO{Abstract: In "... to easily deploy...", "easily" is subjective,
  please don't use. Otherwise, good abstract.}

\section{Introduction}

Cloud Computing can be defined as an abstraction of services from
infrastructures(i.e hardware) \TODO{You need a space between a word
  and the parenthesis that comes after it.}, platforms and
applications(i.e software) by virtualization of resources
\cite{elastic-beanstalk}. The different form of cloud computing
services include IaaS, PaaS, and SaaS which stands for (Infrastructure
as a Service) \TE \TODO{What are these technologies and why are you
  talking about them? You should only mention another technology if it
  helps you explain what Amazon Elastic Beanstalk is. If you do
  mention it, you need to explain what it does and why it's useful.},
Platform as a Service, and Software as a service respectively. With
Cloud Computing, organizations can consume shared computing and
storage resources rather than building, operating and improving
infrastructure on their own and it can also enable organizations to
obtain a flexible, secure and cost-effective Infrastructure. \TODO{Out
  of scope: You don't need a whole paragraph on what cloud computing
  is. Focus on Amazon Elastic Beanstalk.}

Amazon Web Services (AWS) is a subsidiary of Amazon.com, which offers
a suite of cloud computing services such as compute power \TE,
database storage, content delivery and other functionalities that make
up an on-demand computing platform. The scaling on IaaS level can be
illustrated with an example of Amazon Elastic Compute
Cloud(AmazonEC2), whereas scaling on PaaS level can be illustrated
with Amazon Web Services Elastic Beanstalk(AWS Elastic
Beanstalk). \TODO{It's not clear why you're talking about scaling in
  this introduction. These first two sentences are largely
  unnecessary. Focus on Beanstalk.} AWS Elastic Beanstalk is one of
the many services provided by the AWS with its functionality to manage
the Infrastructure.Elastic beanstalk provides a quick deployment and
management of the applications on the Cloud as it automatically
handles the details of capacity provisioning, load balancing, scaling,
and application health monitoring. Elastic beanstalk uses highly
reliable and scalable services \cite{elastic-beanstalk-2}.

\section{Need for AWS Elastic Beanstalk}

Cloud Computing shifts the location of resources to the cloud to
reduce the cost associated with over-provisioning, under-provisioning,
and under-utilization. When more than required resources are
available, is called over-provisioning \cite{cloudcomputing}. When the
resources are not used adequately, it is known as
under-utilization. And \GE when the resources available are not enough is
called under-provisioning. Cloud Computing should also reduce the time
required to provision the resources with the variable workload on the
server so that the applications can be quickly scaled up and down with
the variable workload \cite{elastic-beanstalk-book}. There scaling of
applications can be achieved by:

\TODO{Out of scope. What does this have to do with Beanstalk? If you
  feel strongly you need to include it, you need to motivate why.}

\subsection{Manual Scaling in Cloud Enviornment}

In traditional applications, scalability is achieved by predicting the
peak loads, then purchasing, setting up and configuring the
infrastructure that could handle this peak load \cite{ManualScaling}. With Manual
Scaling, the resources are provisioned at the deployment time and the
application servers are added to infrastructure manually, thus there is
high latency. Due to these issues, the average time the applications
are provisioned is long. There is also an issue of Manual
Monitoring of the resources allocated.

\TODO{Out of scope. What does this have to do with Beanstalk?}

\subsection{Semi-Manual Scaling in Cloud Enviornment}

The resources provided for the Inftrastruare are virtualized in the cloud
environment and thus this virtualization enables the elasticity. In
particular, in cloud environments, resources are provisioned
dynamically (i.e. at runtime), automatically (i.e. without user
intervention), infinitely and almost immediately (i.e. within minutes
and not hours, days, weeks or months like in traditional
environments) \cite{elastic-beanstalk}. Thus, the mean time until when
the resources are provisioned are short.  However, manual monitoring
of resources is still necessary. In particular, users are forced to
make a tradeoff between requesting more resources to avoid
under-provisioning and requesting fewer resources to avoid
over-provisioning and under-utilization. Since resources are
provisioned by request, the problem of the unavailability of an
application at peak loads is not completely eliminated.

\TODO{Out of scope. You need to focus on Beanstalk.}

\subsection{Automatic Scaling in Cloud Enviornment}

Automatic Scaling enables users to closely follow the workload curve
of their application, by provisioning resources on demand. The user
owns the choice that the number of resources these applications are
using increases automatically during the time when the demand of
resources are high to handle the peak load and also automatically
decreases when the minimal resources are needed, to minimize the cost
so that the user only pays for what they used. Automatic Scaling also
predicts the peak load that the applications may require in the future
and provisions these required resources in advance, proving the
elasticity of the cloud  \cite{elastic-beanstalk}.

\TODO{Out of scope. You need to focus on Beanstalk.}

\section{Elastic Beanstalk services}

Elastic Beanstalk supports applications developed in Java, PHP, NET,
Node.js, Python, and Ruby, and also in different container types for
each language. A container defines the infrastructure and software
stack to be used for a given environment. When an application is
deployed, Elastic Beanstalk provisions one or moreAWS resources, such
as Amazon EC2 Instances \cite{elastic-beanstalk-book}. The software
stack that runs on the instances depends on the container type, where
two container types are supported by the Elastic Beanstalk Node.js: a
32-bit Amazon Linux Image and a 64-bit Amazon Linux Image. Where \GE each
of them runs the Software stack tailored to the hosted Node.js
application. Amazon Elastic Beanstalk can be interacted using the AWS
Management Console, the AWS Command Line Interface(AWS CLI), or a
high-level CLI designed for Elastic BeanStalk
\cite{elastic-beanstalk-book}.

Amazon Elastic Beanstalk provides the Automatic Scaling in cloud
Environments.  For Automatic Scaling, it uses the following Amazon Web
Services:

\subsection{AWS Management Console}

AWS Management Console is a browser-based graphical user
interface(GUI) for Amazon Web Services. It allows users to configure
an automatic scaling mechanism of AWS Elastic Beanstalk as well as
other services of AWS. From the Management console, the user can decide
about how many instances does the application require. When the
application must be scaled up and down \cite{elastic-beanstalk}.

\subsection{Elastic Load Balancing}
 
Elastic Load Balancing enables the load balancer which automatically
distributes the incoming application traffic across all running
instances in the auto-scaling group based on metrics like request count
and latency tracked by Amazon Cloud Watch. If an instance is
terminated, the load balancer will not route requests to this instance
anymore. Rather, it will distribute the requests across the remaining
instances.
    
Elastic Load Balancing also monitors the availability of an
application, by checking its "health" periodically (e.g. every five
minutes). If this check fails, AWS Elastic Beanstalk will execute
further tests to detect the cause of the failure
\cite{elastic-loadbalancing}. In particular, it checks if the load
balancer and the auto-scaling group are existing.  In addition, it
checks if at least one instance is running in the auto-scaling
group. Depending on the test results, AWS Elastic Beanstalk changes
the health status of the application.
    
The different colors respond to the status of the application. Green
indicates that the application responded within a minute, Yellow
indicates that the application has not responded in the last 5
minutes, Red indicates that application has not responded in the last
5 minutes or some other problem may have been detected by AWS Elastic
Beanstalk. Gray indicates that the status of the application is
unknown as of now \cite{elastic-loadbalancing}.

\TODO{Like other sections, this is out of scope. You spend too much
  space going into details that don't directly have to do with
  Beanstalk. You either need to motivate why you are including this in
  the paper, or you need to remove it or significantly shorten it.}

\subsection{Auto Scaling}

Auto Scaling automatically launches and terminates instances based on
metrics like CPU and RAM utilization of the application and are
tracked by Amazon CloudWatch and thresholds called triggers. Whenever
a metric crosses a threshold, a trigger is fired to initiate automatic
scaling.  For example, a new instance will be launched and registered
at the load balancer if the average CPU utilization of all running
instances exceeds an upper threshold
    
Auto Scaling also provides fault tolerance. If an instance reaches an
unhealthy status or terminates unexpectedly, Auto Scaling will
compensate this and launch a new instance instead, thus assuring that
the specified minimum number of instances are running constantly.

\TODO{Not only is his out of scope, you already discussed scaling at
  length earlier in your paper.}

\subsection{Amazon Cloud Watch}

Amazon CloudWatch is a component of Amazon Web Services (AWS) that
provides monitoring for AWS resources and the customer applications
running on the Amazon infrastructure \cite{elastic-cloudwatch}  It tracks and stores
per-instance metrics, including request count and latency, CPU and RAM
utilization.  Once stored, the metrics can be visualized an API,
command-line tools, one of the AWS SDK (software development kits) or
the AWS Management Console. With AWS Management Console, users can
get real-time visibility into the utilization of each of the instances
in an auto-scaling group via the graphs and chart provided, and can
easily and quickly detect the over-provisioning, under-utilization or
under-provisioning.

\TODO{Out of scope. What you said about Amazon Cloud Watch earlier in the paper is enough.}

\section{Conclusion}

In traditional (i.e. non-cloud) environments, overprovisioning and
under-utilization can hardly be avoided \cite{cloudcomuting}. There is
an observation that in many companies the average utilization of
application servers ranges from 5 to 20 percent, meaning that many
resources like CPU and RAM are idle at peak times
\cite{cloudcomputing-2}.  On the other hand, if the companies shrink
their infrastructures to reduce over-provisioning and
under-utilization, the risk of under-provisioning will increase. While
the costs of over-provisioning and under-utilization can easily be
calculated, the costs of under-provisioning are more difficult to
calculate because under-provisioning can lead to a no peak times
\cite{cloudcomputing-2}.

Other platforms like Google App Engine \cite{google-appengine} also
let users have their applications to automatically scale both up and
down according to demand but with even more restrictions on how users
should develop their applications. Wheras, with AWS Elastic Beanstalk
there is more control with the user. A downside of AWS Elastic
Beanstalk is that currently, it does not provide any web service to
predict demand for the near future. Theoretically, the statistical
usage data of the last few months or years could be used to predict
time intervals during which more or fewer resources are needed.

\TODO{This conclusion barely mentions Beanstalk.}

% Bibliography

\bibliography{references}
 


\end{document}

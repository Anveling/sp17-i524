\documentclass[9pt,twocolumn,twoside]{../../styles/osajnl}
\usepackage{fancyvrb}
\journal{i524} 

\title{Apache Ranger}

\author[1,*, +]{Avadhoot Agasti}

\affil[1]{School of Informatics and Computing, Bloomington, IN 47408, U.S.A.}

\affil[*]{Corresponding authors: aagasti@indiana.edu}

\affil[+]{HID - SL-IO-3000}

\dates{paper2, \today}

\ociscodes{Apache Ranger, LDAP, Active Directory, Apache Knox, Apache Atlas,
Apache Hive, Apache Hadoop, Yarn, Apache HBase, Apache Storm, Apache Kafka,
Data Lake, Apache Sentry}

% replace this with your url in github/gitlab
\doi{\url{https://github.com/avadhoot-agasti/sp17-i524/tree/master/paper2/S17-IO-3000/report.pdf}}


\begin{abstract}
Apache Hadoop provides various data storage, data access and data processing
services. Apache Ranger is part of the Hadoop eco-system. Apache Ranger
provides capability to perform security administration tasks for storage,
access and processing of data in Hadoop. Using Ranger, Hadoop administrator
can perform security administration tasks using a central UI or Restful web
services. He can define policies which enable users/user-groups to perform
specific action using Hadoop components and tools. Ranger provides role based
 access control for datasets on Hadoop at column and row level. Ranger also
 provides centralized auditing of user access and security related
 administrative actions.\newline
\end{abstract}

\setboolean{displaycopyright}{true}

\begin{document}

\maketitle

\section{Introduction}

Apache Ranger is open source software project designed to provide centralized
 security services to various components of Apache Hadoop. Apache Hadoop
 provides various mechanism to store, process and access the data. Each
 Apache tool has its own security mechanism. This increases administrative
 overhead and is also error prone. Apache Ranger fills this gap to provide
 a central security and auditing mechanism for various Hadoop components. Using
  Ranger, Hadoop administrator can perform security administration tasks
  using a central UI or Restful web services. He can define policies which
  enable users/user-groups to perform specific action using Hadoop components
   and tools. Ranger provides role based access control for datasets on
   Hadoop at column and row level. Ranger also provides centralized auditing
   of user access and security related administrative actions.

\section{ARCHITECTURE OVERVIEW}

\cite{www-ranger-architecture} describes the important components of Ranger as
 explained below:

\subsection{Ranger Admin Portal}

Ranger Admin Portal is the main interaction point for the user. Using Admin
Portal, user can define policies. The policies are stored in Policy Database.
 The Policies are polled by various plugins. The Admin Portal also collects
 the audit data from plugins and stores in HDFS or in a relational database.

\subsection{Ranger Plugins}

Plugins are Java Programs, which are invoked as part of the cluster component.
 For example, the ranger-hive plugin is embedded as part of Hive Server2. The
  plugins cache the policies, and intercept the user request and evaluates it
   against the policies. Plugins also collect the audit data for that
   specific component and send to Admin Portal.

\subsection{User group sync}

While Ranger provides authorization or access control mechanism, it needs to
know the users and the groups. Ranger integrates with Unix users management or
LDAP or Active Directory to fetch the users and groups. The User group sync
component is responsible for this integration.

\section{HADOOP COMPONENTS SUPPORTED BY RANGER}

Ranger supports auditing and authorization for following Hadoop components
\cite{www-ranger-faq}.

\subsection{Apache Hadoop and HDFS}

Apache Ranger provides plugin for Hadoop, which helps in enforcing data access
policies. The HDFS plugin works with Name Node to check if the user's access
request to a file on HDFS is valid or not.

\subsection{Apache Hive}

Apache Hive provides SQL interface on top of the data stored in HDFS. Apache
Hive supports two types of authorization - Storage based authorization and
SQL standard authorization. Ranger provides centralized authorization
interface for Hive which provides granular access control at table and column
 level. Ranger implements a plugin which is part of Hive Server2.

\subsection{Apache HBase}
Apache HBase is NoSQL database implemented on top of Hadoop and HDFS. Ranger
provides coprocessor plugin for HBase, which performs authorization checks and
 audit log collections.

\subsection{Apache Storm}
Ranger provides plugin to Nimbus server which helps in performing the
security authorization on Apache Storm.

\subsection{Apache Knox}
Apache Knox provides service level authorization for users and groups. Ranger
 provides plugin for Knox using which, administration of policies can be
 supported. The audit over Knox data enables user to perform detailed
 analysis of who and when accessed Knox.

\subsection{Apache Solr}
Solr provides free text search capabilities on top of Hadoop. Ranger is
useful to protect Solr collections from unauthorized usage.

\subsection{Apache kafka}
Ranger can manager access control on Kafka topics. Policies can be
implemented to control which users can write to a Kafka topic and which users
can read from a Kafka topic.

\subsection{Yarn}
Yarn is resource management layer for Hadoop. Administrators can setup queues
 in Yarn and then allocate users and resources per queue basis. Policies can
 be defined in Ranger to define who can write to various Yarn queues.

\section{IMPORTANT FEATURES OF RANGER}
The blog article \cite{www-ranger-key-features} explains the 2 important
features of Apache Ranger.

\subsection{Dynamic Column Masking}
Dynamic data masking at column level is an important feature of Apache Ranger
. Using this feature, the administrator can setup data masking policy. The
data masking makes sure that only authorized users can see the actual data
while other users will see the masked data. Since the masked data is format
preserving, they can continue their work without getting access to the actual
 sensitive data. For example, the application developers can use masked data
 to develop the application whereas when the application is actually
 deployed, it will show actual data to the authorized user. Similarly, a
 security administrator may chose to mask credit card number when it is
 displayed to a service agent.

\subsection{Row Level Filtering}
The data authorization is typically required at column level as well as at row
level. For example, in an organization which is geographically distributed in
 many locations, the security administrator may want to give access of a data
  from a specific location to the specific user. In other example, a
  hospital data security administrator may want to allow doctors to see
  only his or her patients. Using Ranger, such row level access control can
  be specified and implemented.

\section{HADOOP DISTRIBUTION SUPPORT}
Ranger can be deployed on top of Apache Hadoop.
\cite{www-ranger-on-apache-hadoop} provides detailed steps of building and
deploying Ranger on top of Apache Hadoop.

Hortonwork Distribution of Hadoop (HDP) supports Ranger deployment using
Ambari. \cite{www-ranger-on-hdp} provides installation, deployment and
configuration steps for Ranger as part of HDP deployment.

Cloudera Hadoop Distribution (CDH) does not support Ranger. According to
\cite{www-ranger-on-cdh}, Ranger is not recommended on CDH and instead Apache
Sentry should be used as central security and audit tool on top of CDH.

\section{USE CASES}
Apache Ranger provides centralized security framework which can be useful in
many use cases as explained below.

\subsection{Data Lake}
\cite{data-lake-whitepaper} explains that storing many types of data in the
same repository is one of the most important feature of Data Lake. With
multiple datasets, the ownership, security and access control of the data
becomes primary concern. Using Apache Ranger, the security administrator can
define fine grain control on the data access.

\subsection{Multi-tenant Deployment of Hadoop}
Hadoop provides ability to store and process data from multiple
tenants. The security framework provided by Apache Ranger can be utilized to
protect the data and resources from un-authorized access.

\section{Apache Ranger and Apache Sentry}
According to \cite{www-5security-blog}, Apache Sentry and Apache Ranger have
many features in common. Apache Sentry (\cite{www-apache-sentry}) provides
role based authorization to data and metadata stored in Hadoop.

\section{Educational Material}
\cite{www-ranger-tutorial} provides tutorial on topics like A)Security
resources B)Auditing C)Securing HDFS, Hive and HBase with Knox and Ranger D)
Using Apache Atlas' Tag based policies with Ranger.
\cite{www-ranger-quickstart} provides step by step guidance on getting latest
code base of Apache Ranger, building and deploying it.

\section{Licensing}
Apache Ranger is available under Apache 2.0 License.

\section{Conclusion}
Apache Ranger is useful to Hadoop Security Administrators since it enables
the granular authorization and access control. It also provides central
security framework to different data storage and access mechanism like Hive,
HBase and Storm. Apache Ranger also provides audit mechanism. With Apache
Ranger, the security can be enhanced for complex Hadoop use cases like Data
Lake.

\section*{Acknowledgements}

The authors thank Prof. Gregor von Laszewski for his technical guidance.

% Bibliography

\bibliography{references}

\end{document}

\documentclass[9pt,twocolumn,twoside]{../../styles/osajnl}
\usepackage{fancyvrb}
\journal{i524} 

\title{An overview of Cisco Intelligent Automation for Cloud}

\author[1,*]{Bhavesh Reddy Merugureddy}

\affil[1]{School of Informatics and Computing, Bloomington, IN 47408, U.S.A.}

\affil[*]{Corresponding authors: bmerugur@umail.iu.edu}

\dates{Paper2, \today}

\ociscodes{Automation, Multitenancy, Integration, Orchestration, Provisioner}

% replace this with your url in github/gitlab
\doi{\url{https://github.com/cloudmesh/sp17-i524/blob/master/paper2/S17-IR-2018/report.pdf}}


\begin{abstract}
Cisco Intelligent automation for cloud is a cloud platform that can deliver services across mixed environments. Its services range from underlying infrastructure to anything-as-a-service and allows the users to evaluate, transform and deploy various IT services. It provides a foundation for organizational transformation by expanding the uses of cloud technology beyond its infrastructure.\newline
\end{abstract}

\setboolean{displaycopyright}{true}

\begin{document}

\maketitle

\section{Introduction}

Cisco Intelligent Automation for Cloud (IAC) provides a framework that
allows users to make use of the cloud services effectively and manage
the cloud beyond Infrastructure-as-a-service (IaaS)
\cite{ciscolive-introduction}.  It can be considered as a unified
cloud platform that can deliver any type of service across mixed
environments \cite{cisco-wikibon}. This leads to an increase in cloud
penetration across different businesses. Its services range from
underlying infrastructure to anything-as-a-service by allowing its
users to evaluate, transform and deploy the IT and business services
in a way they desire. Cisco Intelligent Automation for Cloud automates
sophisticated data center from a single self-service portal which is
beyond the provision of virtual machines. It creates a catalog of
standardized service offerings, thereby implementing policy-based
controls. It also provides resource management across different
aspects of IT infrastructure such as network, virtualization, storage,
compute and applications.

\section{Components}

Cisco Intelligent Automation for Cloud consists of major interface and
automation components. In other words, it provides an integrated stack
of core elements namely Cisco Cloud Portal, Cisco Process
Orchestrator, Cisco Process Orchestrator Integration Framework, Cisco
Server Provisioner, and Cloud Automation Packs \cite{cisco-datasheet}.

\subsection{Cisco Cloud Portal}

Cisco Cloud Portal is a web-based service portal that helps users to
order and manage services. It provides a configurable interface for
different roles and departments which include managers, administrators
and consumers \cite{cisco-datasheet}. This provides an access to a
catalog of on-demand IT resources. Service requests for resources
running on cross-platform infrastructure can be managed effectively by
Cisco Cloud Portal. It tracks and manages lifecycle of each service
from the initial request to withdrawal. The modules which constitute
Cisco Cloud Portal are Cisco Portal Manager, Cisco Service Catalog,
Cisco Request Center and Cisco Service Connector.

Cisco Portal Manager combines data from multiple sources providing a
highly flexible portal interface. It manages the interface and
increases user satisfaction. It provides a drag-and-drop facility on
the interface making it easier for the users to create their own
portal views which ensures flexibility
\cite{cloudportal-datasheet}. Depending on the user requests, Cisco
Service Catalog provides a controlled access to IT resources through
standardized service options. It makes use of reusable components and
tools for publishing services in the portal. Cisco Request Center
provides lifecycle management and request management for the
infrastructure services. To ensure data center management, it
maintains policy-based controls and reduces the cycle time for the
ordering, approval and provisioning process. It uses advanced methods
for simplifying the ordering process. It also helps in streamlining
end-to-end service delivery cycle time. Cisco Service Connector is a
platform that can be used for integrating with third-party
systems. This supports a heterogeneous data center environment by
providing adaptors for integrating with automated provisioning
systems.

\subsection{Cisco Process Orchestrator}

Cisco Process Orchestrator is a component of Cisco IAC responsible for
automation of service management and assurance instantiation. It is an
orchestration engine that provides an automation design studio and a
reporting and analytics module. It is useful in automating and
standardizing IT processes in heterogeneous environments
\cite{cisco-orchestrator}. It considers IT automation as a
service-oriented approach and focuses more on services. It allows
users to deploy services by defining new instances in real time. Cisco
Process Orchestrator acts as a foundation for building application,
network and data center oriented solutions. It includes auditing and
extensive reporting and provides workspaces for developers and
administrators making it easier for the stakeholders to manage
services.

The primary function of Cisco Process Orchestrator is to provide
automation through integration with domain managers and tools in the
environment \cite{orchestrator-datasheet}. Automation Packs and
Adapters are the features used in the integration. Some of the
services include event correlation, application provisioning and event
application provisioning. It usually receives events requiring further
analysis. It also includes security tools, configuration tools, change
management systems, visualization management tools, provisioning
systems and service desks.

\subsection{Cisco Process Orchestrator Integration Framework}

Cisco Process Orchestrator Integration Framework is responsible for
integrating Cisco IAC with any data center element in the
environment. It connects IT service management tools into streamlined
and automated processes by making use of field-built
integration. VMware, SAP, Oracle DB, Remedy and Windows are some of
the available integrations. Field integration and automation are
carried out by the design studio which provides web services, database
access and command-line interface.

\subsection{Cisco Server Provisioner}

Cisco Server Provisioner is a software application used for deploying
systems with Linux, Windows and Hypervisors from bare metal in IT
organizations and data centers \cite{cisco-provisioner}. It acts as an
application for native and remote installations on physical and
virtual servers. It provides imaging component for OS and
Hypervisor. The Cisco Server Provisioner can be considered as a server
suite that consists of Bare Metal Provisioning Payloads and Bare Metal
Imaging Payloads which provide GUI and API interfaces including remote
file copying and remote troubleshooting. Provisioner runs on a
dedicated system which does not run any other application. Some of the
benefits of the Provisioner include reduced deployment time and
increased utilization of systems.

\subsection{Cloud Automation Packs}

Cloud Automation Packs are the workflows useful for complex computing
tasks. The tasks include automation of core activities that cover
various domains, Cisco Server Provisioner task automation, VMware task
automation and Cisco UCS Manager task automation. Automation Packs
provide a set of target groups, variables, configurations and process
definitions required for defining automated IT processes. Automation
Packs combine with Adaptors to enable integrations. Integration with
IT element is carried out through a combination of automation content
from an Automation Pack and a set of Adaptors. Some of the integration
scenarios include Command Line Interface invocations, Web Service
integration, Messaging integration and Scripting support. Automation
content can leverage Adaptors to enable these scenarios. Automation
Packs can be used to build, pack, update and ship the integrations.

\section{Functions}

Cisco Intelligent Automation for Cloud provides a platform for
designing, deploying and operating a cloud infrastructure in a public,
private or a hybrid model. It supports various cloud management
activities including setup and design, system operations, reporting
and analytics.

\subsection{Self-Service Interface}

Self-Service Interface is a web-based interface which allows the users
to view the service catalog according to their roles and other access
controls. It provides dynamic forms by which users can provide
configuration details and order services. It also allows the users to
track order status, manage and modify placed orders and view the usage
and consumption.

\subsection{Service Delivery Automation}

After the approval of the placed orders, Service Delivery Automation
takes place to orchestrate the configuration and provision of
resources like compute, network, storage and supporting services such
as firewall, disaster recovery and load balancing
\cite{cisco-datasheet}. The automated provisioning provides
consumption tracking and integration into metering and billing
systems. The automated processes then orchestrate the configuration
updates and allows the service information updates to be sent back to
the system management tools and web-based portal.

\subsection{Operational Process Automation}

Operational Process Automation coordinates the operational tasks for
cloud management which include service-level management, alerting,
reporting, capacity planning, performance management, user management
and maintenance checks. It provides user administration capability to
control user roles and identity, placing the users securely isolated
from each other. Systems incidents can be managed by the users through
alert management feature and all the processes and results can be
tracked and reported by the reporting functionality. Operational
Process Automation consists of an Automation Control Center which is a
console for viewing and controlling automated processes.

\subsection{Network Automation}

Cisco IAC allows a manual pre-provisioning of network layer with the
increasing amount of data being placed in cloud. This is carried out
by Network Automation. Network Automation enables deployment of
network services through the Self-Service Portal with a single order
\cite{cisco-mondora}. It facilitates dynamic installation of virtual
network devices with onboarding users and creates network topologies
for the users based on the applications they use.

\subsection{Cloud Governance}

Through Cloud Governance, Cisco IAC delivers a set of measures for
tracking business-oriented metrics. Cloud Governance provides
portfolio management across different cloud environments. It allows
organizations to establish limits ahead of time. Financial granularity
is achieved by these consumption limits.

\subsection{Advanced Multitenancy}

Advanced Multitenancy allows multiple users to securely reside in a
shared environment by providing isolated containers. User management
work is offloaded from the cloud provider as the users are controlled
by the cloud administrators. Multitenancy sets service pricing per
user and accordingly, the services can be enabled or disabled per
organization or user. It provides onboarding, modification and
offboarding of users and enables secure containers by instantiating
network devices.

\subsection{Multicloud Management}

By Multicloud Management, service providers can tailor their services
to specific project needs and specific functions of the hypervisor
platforms. For example, Cisco IAC supports two infrastructure layers
namely Cisco UCS Director and OpenStack \cite{cisco-mondora}. It
allows administrators to manage network services and virtual machines
by integrating with Havana and Icehouse. This allows the Cisco UCS
Director users to manage Microsoft System Center Virtual Machine
Manager.

\subsection{Resource Management}

Cisco IAC Resource Management orchestrates resource-level operations
across different hypervisors such as Hyper-V, Xen or VMware, compute
resources such as Cisco UCS, network resources and storage resources
such as NetApp. This is done by orchestrating the requests to domain
resource managers. It provides maintenance and replacement of
units. Capacity management is provided by automated capacity
utilization checks, trending reports and alerts. The usage and user
quota are managed by automated monitoring and metering of user
accounts.

\subsection{Lifecycle Management}

Lifecycle Management is responsible for creation of service
definitions which include selection parameters, business and technical
processing flows, pricing options and design descriptions. It also
involves management of a service model and underlying automation
design for managing a service. Automation design is managed by
creating workflows to automate service provisioning, modification,
decommissioning and upgrades. The designs are modified by
point-and-click-tools instead of custom programming. Lifecycle
Management tracks all aspects of services that are running which
includes business and project information.

\section{Deployment}

Cisco Intelligent Automation for Cloud is deployed as a software
solution. For planning, preparation, design, implementation and
optimization of cloud services, it is deployed along with services
engagement. The services engagement involves creation of automation
workflows and development of a cloud strategy. It focuses on service
capabilities for the software deployment.

\section{Conclusion}

Cisco Intelligent Automation for Cloud is a framework useful in
expanding cloud responsiveness and flexibility. It provides services
beyond provisioning virtual machines. It delivers various services in
a self-service manner across mixed environments. The cloud management
in Cisco Intelligent Automation for Cloud allows users to evaluate,
procure and deploy IT services according to the needs. Self-Service
Portal, Service Delivery Automation, Network Automation, Resource
Management, Provisioning, Advanced Multitenancy, Portfolio Management
and Lifecycle Management are the elements responsible for the cloud
management in Cisco Intelligent Automation for Cloud.

\section{Acknowledgements}

The author thanks Professor Gregor Von Lazewski and all the AIs of Big
Data class for the guidance and technnical support.


% Bibliography

\bibliography{references}
 


\end{document}
